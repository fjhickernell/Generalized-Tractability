\documentclass{article}
\usepackage{amsmath,amssymb, amsthm, booktabs, mathtools}
\usepackage[utf8]{inputenc}
\input{FJHDef.tex}

\title{Generalized Tractability}
\author{Fred J. Hickernell, Peter Kritzer, Onyekachi Osisiogu, and \\ Henryk Wo\'zniakwoski}
\date{\today}

%\DeclareMathOperator{\comp}{comp}
\DeclareMathOperator{\SOL}{SOL}
\DeclareMathOperator{\APP}{APP}

%\theoremstyle{theorem}
\newtheorem{theorem}{Theorem}
\theoremstyle{definition}
\newtheorem{definition}{Definition}


\begin{document}

\maketitle

\section{Introduction}

Let $\cf_d$ and $\cg_d$ Hilbert spaces, and $\SOL_d : \cf_d \to \cg_d$ be a linear solution operator with adjoint $\SOL_d^*$ such that $\SOL_d^*\SOL_d : \cf_d \to \cf_d$ has eigenvalues and orthonormal eigenvectors  
\[
\lambda_{\max}^2 \ge \lambda_{1,d}^2 \ge \lambda_{2,d}^2 \ge \cdots, \qquad u_{1,d}, u_{2,d}, \ldots, 
\]
This means that the non-negative $\lambda_{i,d}$ are the singular values of the $\SOL_d$. The goal is to find an approximate solution, $\APP_d:\cb_d \times (0,\infty) \to \cg_d$ satisfying 
\[
\norm[\cg_d]{\SOL_d(f) - \APP_d(f,\varepsilon)} \le \varepsilon,
\]
where $\cb_d$ is the unit ball in $\cf_d$ and $\APP_d(f,\varepsilon)$ is allowed to depend on arbitrary linear functionals.  In this case the optimal solution is 
\[
\APP_d(f,\varepsilon) = \sum_{i=1}^n \SOL(u_i) \ip[\cf_d]{f}{u_i},
\]
and the complexity of our linear problem is
\[
\comp(\varepsilon, d) = \min \{n \in \natzero : \lambda_{n+1, d} \le \varepsilon\}.
\]

Certain kinds of common notions of tractability are 
\begin{equation*}
    \begin{array}{r@{\qquad}c@{\qquad}c}
    & \comp(\varepsilon, d)\\
    \toprule
    \text{strong algebraic tractability} & \Theta(\varepsilon^{-p})\\
    \text{ algebraic tractability} & \Theta(\varepsilon^{-p}d^{q})\\
    \text{strong exponential tractability} &  \Theta\bigl([\log(1 + \varepsilon^{-1})]^p\bigr)\\
    \text{ exponential tractability} & \Theta\bigl([\log(1 + \varepsilon^{-1})]^p  d^{q}\bigr)
    \end{array}
\end{equation*}
Let let $\phi : (0,\infty) \times (0,\infty] \to (0,\infty)$ and $\psi : \naturals  \times (0,\Delta) \to (0,\infty)$ be positive-valued functions that are strictly increasing in both arguments, e.g., 
\[
(x,p) \mapsto x^{p}, \qquad (x,p) \mapsto [\log(1+x)]^p.
\]

All the examples of tractability above are of the form $\comp(\varepsilon, d) = \Theta \bigl( \phi(\varepsilon^{-1},p) \psi(d,q)\bigr)$.  Note, we define the $\Theta$ notation as follows
\begin{equation*}
     g(\varepsilon,d) = \Theta\bigl(\phi(\varepsilon^{-1},p) \bigr) \text{ means } 
    \lim_{\varepsilon \to 0} \frac{g(\varepsilon,d)}{\phi(\varepsilon^{-1},p')}
           \begin{cases} < \infty, & p'> p,  \\ 
          =\infty, & p' < p.\end{cases}
\end{equation*}
\begin{multline*}
     g(\varepsilon,d) = \Theta\bigl(\phi(\varepsilon^{-1},p) \psi(d,q) \bigr) \text{ means } \\
    \lim_{\substack{\varepsilon \to 0 \\ d \to \infty}} \frac{g(\varepsilon,d)}{\phi(\varepsilon^{-1},p') \psi(d,q')}
          \begin{cases} < \infty, & p'> p, \  q' > q,  \\ 
          =\infty, & p' < p, \ q' < q.\end{cases}
\end{multline*}


\begin{definition}
If there exist the functions $\phi$ and $\psi$ satisfying the conditions listed above and there exist positive numbers $p$ and $q$ with
\[
\comp(\varepsilon, d)  = \Theta\bigl(\phi(\varepsilon^{-1},p) \psi(d,q) \bigr),
\]
then we call a problem $\phi$-$\psi$-tractable with exponents $p$ and $q$.  If there exists a function there $\phi$ satisfying the conditions above and a positive numbers $p$
\[
\comp(\varepsilon, d)  = \Theta\bigl(\phi(\varepsilon^{-1},p) \bigr)
\]
then we call a problem strongly $\phi$-tractable with expontent $p$
\end{definition}



\section{Main Result}
\begin{theorem} A problem is $\phi$-$\psi$-tractable iff there exists a  $C : (0,\infty)^2 \to \naturals$
\begin{equation} \label{eq:tractiff}
    \sup_{d \in \naturals} \frac{1}{\psi(d,q)} \sum_{i = C(p,q)}^\infty \frac{1}{\phi(\lambda_{i, d}^{-1/2},p)} \le 1 \qquad \forall p,q \in (0,\infty)
\end{equation}
\end{theorem}
\begin{proof}
Fix $\delta_1 \in (0,\Delta]$.  Because $\phi$ is increasing in its arguments, it follows that the complexity may be equivalently expressed as 
\begin{align*}
    \comp(\varepsilon, d) &= \min \{n \in \natzero : \lambda_{n+1, d} \le \varepsilon^2\} \\
    &= \min \biggl\{n \in \natzero : \frac{1}{\phi(\lambda_{n+1, d}^{-1/2},\delta_1)} \le \frac{1}{\phi(\varepsilon^{-1},\delta_1)} \biggr\}.
\end{align*}

Since the $\lambda_{i,d}$ are non-increasing it follows that the $\phi(\lambda_{i, d}^{-1/2},\delta_1)$ are also non-increasing.  In particular, for all $n \in \naturals$,
\begin{align*}
    \MoveEqLeft{\lambda_{n+1,d} \le \lambda_{n,d} \le \cdots \le \lambda_{1,d}} \\
    & \implies \frac{1}{\phi(\lambda_{n+1, d}^{-1/2},\delta_1)} \le \frac{1}{\phi(\lambda_{n, d}^{-1/2},\delta_1)} \le \cdots \le \frac{1}{\phi(\lambda_{1, d}^{-1/2},\delta_1)} \\
    & \implies \frac{1}{\phi(\lambda_{n+1, d}^{-1/2},\delta_1)} 
    \le \frac 1n \sum_{i=1}^n  \frac{1}{\phi(\lambda_{i, d}^{-1/2},\delta_1)} 
    \le \frac 1n \sum_{i=1}^\infty  \frac{1}{\phi(\lambda_{i, d}^{-1/2},\delta_1)}.
\end{align*}
Thus, we can conclude that 
\begin{align*}
    n \ge \phi(\varepsilon^{-1},\delta_1) \sum_{i=1}^\infty \frac{1}{\phi(\lambda_{i, d}^{-1/2},\delta_1)}
    & \implies 
  \frac 1n \sum_{i=1}^\infty \frac{1}{\phi(\lambda_{i, d}^{-1/2},\delta_1)} \le  \frac{1}{\phi(\varepsilon^{-1},\delta_1)} \\
   & \implies   \frac{1}{\phi(\lambda_{n+1, d}^{-1/2},\delta_1)} \le \frac{1}{\phi(\varepsilon^{-1},\delta_1)}.
\end{align*}
This implies an upper bound on the complexity of
\begin{align*}
       \MoveEqLeft{\comp(\varepsilon,d)} \\
       &\le 1 + \phi(\varepsilon^{-1},\delta_1) \sum_{i=1}^\infty \frac{1}{\phi(\lambda_{i, d}^{-1/2},\delta_1)} \\
       & = 1 + \phi(\varepsilon^{-1},\delta_1) \left [ \sum_{i=1}^{C_3(\delta_1,\delta_2)-1} \frac{1}{\phi(\lambda_{i, d}^{-1/2},\delta_1)}  + \sum_{i=C_3(\delta_1,\delta_2)}^\infty \frac{1}{\phi(\lambda_{i, d}^{-1/2},\delta_1)} \right ] \\
       & \le \underbrace{1+  \phi(\varepsilon^{-1},\delta_1) \frac{C_3(\delta_1,\delta_2)-1}{\phi(\lambda_{\max}^{-1/2},\delta_1)}}_{C_1(\delta_1,\delta_2)}
       +  \phi(\varepsilon^{-1},\delta_1) \psi(d,\delta_2) 
        \qquad \text{by \eqref{eq:tractiff}} \\
\end{align*}
This verifies the sufficient condition.



The proof is done.
\end{proof}

\end{document}
