\documentclass{article}
\usepackage{amsmath,amssymb, amsthm, booktabs, mathtools}
\usepackage[utf8]{inputenc}
\input{FJHDef.tex}

\title{Generalized Tractability}
\author{Fred J. Hickernell, Peter Kritzer, Onyekachi Osisiogu, and \\ Henryk Wo\'zniakwoski}
\date{\today}

%\DeclareMathOperator{\comp}{comp}
\DeclareMathOperator{\SOL}{SOL}
\DeclareMathOperator{\APP}{APP}

%\theoremstyle{theorem}
\newtheorem{theorem}{Theorem}
\theoremstyle{definition}
\newtheorem{definition}{Definition}


\begin{document}

\maketitle

\section{Introduction}

Let $\cf_d$ and $\cg_d$ Hilbert spaces, and $\SOL_d : \cf_d \to \cg_d$ be a linear solution operator with adjoint $\SOL_d^*$ such that $\SOL_d^*\SOL_d : \cf_d \to \cf_d$ has eigenvalues and orthonormal eigenvectors  
\[
\lambda_{\max}^2 \ge \lambda_{1,d}^2 \ge \lambda_{2,d}^2 \ge \cdots, \qquad u_{1,d}, u_{2,d}, \ldots, 
\]
This means that the $\lambda_{i,d}$ are the singular values of the $\SOL_d$. The goal is to find an approximate solution, $\APP_d:\cb_d \times (0,\infty) \to \cg_d$ satisfying 
\[
\norm[\cg_d]{\SOL_d(f) - \APP_d(f,\varepsilon)} \le \varepsilon,
\]
where $\cb_d$ is the unit ball in $\cf_d$ and $\APP_d(f,\varepsilon)$ is allowed to depend on arbitrary linear functionals.  In this case the optimal solution is 
\[
\APP_d(f,\varepsilon) = \sum_{i=1}^n \SOL(u_i) \ip[\cf_d]{f}{u_i},
\]
and the complexity of our linear problem is
\[
\comp(\varepsilon, d) = \min \{n \in \natzero : \lambda_{n+1, d} \le \varepsilon^2\}.
\]

Certain kinds of common notions of tractability are 
\begin{equation*}
    \begin{array}{r@{\qquad}c@{\qquad}c}
    & \comp(\varepsilon, d)\\
    \toprule
    \text{strong algebraic tractability} & \Theta(\varepsilon^{-p})\\
    \text{ algebraic tractability} & \Theta(\varepsilon^{-p}d^{q})\\
    \text{strong exponential tractability} & \log(1 + \varepsilon^{-p})\\
    \text{ exponential tractability} &(p+\delta) \log(1 + x) &  x^{q+\delta} \\
    & 
    \end{array}
\end{equation*}

Let let $\phi : (0,\infty) \times [0,\Delta] \to (0,\infty)$ and $\{\psi : \naturals  \times [0,\Delta] \to (0,\infty)\}$ be positive functions that are increasing in both arguments, e.g., 
\[
(x,\delta) \mapsto x^{p + \delta} \ (p >0), \qquad (x,\delta) \mapsto (1 + \delta) \log(1+x)
\]
which satisfy
\begin{equation*}
    \lim_{\delta \downarrow 0} \phi(x,\delta) = \phi(x,0) =: \phi(x), \qquad 
    \lim_{\delta \downarrow 0} \psi(x,\delta) = \psi(x,0) =: \psi(x).
\end{equation*}


\begin{definition}
If for all $\delta_1, \delta_2 \in (0,\Delta]$ there exist the functions $\phi$ and $\psi$ satisfying the conditions listed above, and there exists $C_1 : (0,\Delta]^2 \to (0,\infty)$  with 
\begin{multline*}
    \comp(\varepsilon, d) \le \phi(\varepsilon^{-1},\delta_1)\psi(d,\delta_2) + C_1(\delta_1, \delta_2) \\
    \forall \delta_1, \delta_2 \in (0,\Delta], \quad  \varepsilon \in (0, 1], \quad \forall d \in \naturals
\end{multline*}
then we call a problem $\phi$-$\psi$-tractable.
\end{definition}

Some specific examples include 
\begin{equation*}
    \begin{array}{r@{\qquad}c@{\qquad}c}
    & \phi(x,\delta) & \psi(x,\delta) \\
    \toprule
    \text{strong algebraic tractability} & C_2(x,\delta) x^{p+\delta} & 1\\
    \text{ algebraic tractability} & C_2(x,\delta) x^{p+\delta} & x^{q+\delta}\\
    \text{strong exponential tractability} &(p+\delta) \log(1 + x) & 1 \\
    \text{ exponential tractability} &(p+\delta) \log(1 + x) &  x^{q+\delta} \\
    & 
    \end{array}
\end{equation*}

\section{Main Result}
\begin{theorem} A problem is $\phi$-$\psi$-tractable iff there exists a  $C_3 : (0,\Delta]^2 \to (0,\infty)$
\begin{equation} \label{eq:tractiff}
    \sup_{d \in \naturals} \frac{1}{\psi(d,\delta_2)} \sum_{i = C_3(\delta_1,\delta_2)}^\infty \frac{1}{\phi(\lambda_{i, d}^{-1/2},\delta_1)} \le 1 \qquad \forall \delta_1, \delta_2 \in (0,\Delta]
\end{equation}
\end{theorem}
\begin{proof}
Fix $\delta_1 \in (0,\Delta]$.  Because $\phi$ is increasing in its arguments, it follows that the complexity may be equivalently expressed as 
\begin{align*}
    \comp(\varepsilon, d) &= \min \{n \in \natzero : \lambda_{n+1, d} \le \varepsilon^2\} \\
    &= \min \biggl\{n \in \natzero : \frac{1}{\phi(\lambda_{n+1, d}^{-1/2},\delta_1)} \le \frac{1}{\phi(\varepsilon^{-1},\delta_1)} \biggr\}.
\end{align*}

Since the $\lambda_{i,d}$ are non-increasing it follows that the $\phi(\lambda_{i, d}^{-1/2},\delta_1)$ are also non-increasing.  In particular, for all $n \in \naturals$,
\begin{align*}
    \MoveEqLeft{\lambda_{n+1,d} \le \lambda_{n,d} \le \cdots \le \lambda_{1,d}} \\
    & \implies \frac{1}{\phi(\lambda_{n+1, d}^{-1/2},\delta_1)} \le \frac{1}{\phi(\lambda_{n, d}^{-1/2},\delta_1)} \le \cdots \le \frac{1}{\phi(\lambda_{1, d}^{-1/2},\delta_1)} \\
    & \implies \frac{1}{\phi(\lambda_{n+1, d}^{-1/2},\delta_1)} 
    \le \frac 1n \sum_{i=1}^n  \frac{1}{\phi(\lambda_{i, d}^{-1/2},\delta_1)} 
    \le \frac 1n \sum_{i=1}^\infty  \frac{1}{\phi(\lambda_{i, d}^{-1/2},\delta_1)}.
\end{align*}
Thus, we can conclude that 
\begin{align*}
    n \ge \phi(\varepsilon^{-1},\delta_1) \sum_{i=1}^\infty \frac{1}{\phi(\lambda_{i, d}^{-1/2},\delta_1)}
    & \implies 
  \frac 1n \sum_{i=1}^\infty \frac{1}{\phi(\lambda_{i, d}^{-1/2},\delta_1)} \le  \frac{1}{\phi(\varepsilon^{-1},\delta_1)} \\
   & \implies   \frac{1}{\phi(\lambda_{n+1, d}^{-1/2},\delta_1)} \le \frac{1}{\phi(\varepsilon^{-1},\delta_1)}.
\end{align*}
This implies an upper bound on the complexity of
\begin{align*}
       \MoveEqLeft{\comp(\varepsilon,d)} \\
       &\le 1 + \phi(\varepsilon^{-1},\delta_1) \sum_{i=1}^\infty \frac{1}{\phi(\lambda_{i, d}^{-1/2},\delta_1)} \\
       & = 1 + \phi(\varepsilon^{-1},\delta_1) \left [ \sum_{i=1}^{C_3(\delta_1,\delta_2)-1} \frac{1}{\phi(\lambda_{i, d}^{-1/2},\delta_1)}  + \sum_{i=C_3(\delta_1,\delta_2)}^\infty \frac{1}{\phi(\lambda_{i, d}^{-1/2},\delta_1)} \right ] \\
       & \le \underbrace{1+  \phi(\varepsilon^{-1},\delta_1) \frac{C_3(\delta_1,\delta_2)-1}{\phi(\lambda_{\max}^{-1/2},\delta_1)}}_{C_1(\delta_1,\delta_2)}
       +  \phi(\varepsilon^{-1},\delta_1) \psi(d,\delta_2) 
        \qquad \text{by \eqref{eq:tractiff}} \\
\end{align*}
This verifies the sufficient condition.



The proof is done.
\end{proof}

\end{document}
