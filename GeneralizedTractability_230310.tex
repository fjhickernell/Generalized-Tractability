\documentclass[11pt,a4paper]{article}
\usepackage{latexsym,amsfonts,amsmath,graphics,mathtools,amssymb,amsthm,booktabs}
\usepackage{epsfig}
\usepackage[notref,notcite]{showkeys}
%\usepackage[active]{srcltx}
\usepackage{enumitem}
\usepackage[usenames,dvipsnames,svgnames,table]{xcolor}

\usepackage{algorithmic}

%\input{common-macros}
%\input{common-macros2}

\DeclareSymbolFont{GreekLetters}{OML}{cmr}{m}{it} %Provide missing letters
\DeclareSymbolFont{UpSfGreekLetters}{U}{cmss}{m}{n} %Provide missing letters
\DeclareMathSymbol{\varrho}{\mathalpha}{GreekLetters}{"25}
\DeclareMathSymbol{\UpSfLambda}{\mathalpha}{UpSfGreekLetters}{"03}
\DeclareMathSymbol{\UpSfSigma}{\mathalpha}{UpSfGreekLetters}{"06}
%\newcommand{\bvec}[1]{\boldsymbol{#1}}
\providecommand{\mathbold}{\boldsymbol}
\newcommand{\bvec}[1]{\mathbold{#1}}
%\newcommand{\bvec}[1]{\text{\boldmath$#1$}}
\newcommand{\avec}[1]{\vec{#1}}
%\renewcommand{\vec}[1] {\text{\boldmath$#1$}}
%\renewcommand{\vec}[1]{\ensuremath{\mathbf{#1}}}
%\newcommand{\vecsym}[1]{\ensuremath{\boldsymbol{#1}}}
\newcommand{\vecsym}[1]{\ensuremath{\mathbold{#1}}}
\def\bbl{\text{\boldmath$\{$}}
\def\bbr{\text{\boldmath$\}$}}
\newcommand{\bbrace}[1]{\bbl #1 \bbr}
\newcommand{\bbbrace}[1]{\mathopen{\pmb{\bigg\{}}#1\mathclose{\pmb{\bigg\}}}}
%\def\bbl{\boldsymbol{\left \{}}
%\def\bbr{\boldsymbol{\right \}}}
\def\betahat{\hat\beta}
%\def\e{\text{e}}
%\def\E{\text{E}}
\newcommand{\dif}{{\rm d}}

\newlength{\overwdth}
\def\overstrike#1{ 
\settowidth{\overwdth}{#1}\makebox[0pt][l]{\rule[0.5ex]{\overwdth}{0.1ex}}#1}

\def\abs#1{\ensuremath{\left \lvert #1 \right \rvert}}
\newcommand{\normabs}[1]{\ensuremath{\lvert #1 \rvert}}
\newcommand{\bigabs}[1]{\ensuremath{\bigl \lvert #1 \bigr \rvert}}
\newcommand{\Bigabs}[1]{\ensuremath{\Bigl \lvert #1 \Bigr \rvert}}
\newcommand{\biggabs}[1]{\ensuremath{\biggl \lvert #1 \biggr \rvert}}
\newcommand{\Biggabs}[1]{\ensuremath{\Biggl \lvert #1 \Biggr \rvert}}
\newcommand{\norm}[2][{}]{\ensuremath{\left \lVert #2 \right \rVert}_{#1}}
\newcommand{\normnorm}[2][{}]{\ensuremath{\lVert #2 \rVert}_{#1}}
\newcommand{\bignorm}[2][{}]{\ensuremath{\bigl \lVert #2 \bigr \rVert}_{#1}}
\newcommand{\Bignorm}[2][{}]{\ensuremath{\Bigl \lVert #2 \Bigr \rVert}_{#1}}
\newcommand{\biggnorm}[2][{}]{\ensuremath{\biggl \lVert #2 \biggr \rVert}_{#1}}
\newcommand{\ip}[3][{}]{\ensuremath{\left \langle #2, #3 \right \rangle_{#1}}}

\newcommand{\bigvecpar}[3]{\ensuremath{\bigl ( #1 \bigr )_{#2}^{#3}}}
\newcommand{\Bigvecpar}[3]{\ensuremath{\Bigl ( #1 \Bigr )_{#2}^{#3}}}
\newcommand{\biggvecpar}[3]{\ensuremath{\biggl ( #1 \biggr )_{#2}^{#3}}}
\newcommand{\bigpar}[1]{\ensuremath{\bigl ( #1 \bigr )}}
\newcommand{\Bigpar}[1]{\ensuremath{\Bigl ( #1 \Bigr )}}
\newcommand{\biggpar}[1]{\ensuremath{\biggl ( #1 \biggr )}}

\newcommand{\IIDsim}{\overset{\textup{IID}}{\sim}}

\DeclareMathOperator{\success}{succ}
\DeclareMathOperator{\sinc}{sinc}
\DeclareMathOperator{\sech}{sech}
\DeclareMathOperator{\csch}{csch}
\DeclareMathOperator{\dist}{dist}
\DeclareMathOperator{\spn}{span}
\DeclareMathOperator{\sgn}{sgn}
\DeclareMathOperator*{\rmse}{rmse}
\DeclareMathOperator{\Prob}{\mathbb{P}}
\DeclareMathOperator{\Ex}{\mathbb{E}}
\DeclareMathOperator{\rank}{rank}
\DeclareMathOperator{\erfc}{erfc}
\DeclareMathOperator{\erf}{erf}
\DeclareMathOperator{\cov}{cov}
\DeclareMathOperator{\cost}{cost}
\DeclareMathOperator{\comp}{comp}
\DeclareMathOperator{\corr}{corr}
\DeclareMathOperator{\diag}{diag}
\DeclareMathOperator{\var}{var}
\DeclareMathOperator{\opt}{opt}
\DeclareMathOperator{\brandnew}{new}
\DeclareMathOperator{\std}{std}
\DeclareMathOperator{\kurt}{kurt}
\DeclareMathOperator{\med}{med}
\DeclareMathOperator{\vol}{vol}
\DeclareMathOperator{\bias}{bias}
\DeclareMathOperator*{\argmax}{argmax}
\DeclareMathOperator*{\argmin}{argmin}
\DeclareMathOperator{\sign}{sign}
\DeclareMathOperator{\spann}{span}
\DeclareMathOperator{\cond}{cond}
\DeclareMathOperator{\trace}{trace}
\DeclareMathOperator{\Si}{Si}
%\DeclareMathOperator{\diag}{diag}
\DeclareMathOperator{\col}{col}
\DeclareMathOperator{\nullspace}{null}
\DeclareMathOperator{\Order}{{\mathcal O}}
%\DeclareMathOperator{\rank}{rank}

\newcommand{\vzero}{\bvec{0}}
\newcommand{\vone}{\bvec{1}}
\newcommand{\vinf}{\bvec{\infty}}
\newcommand{\va}{\bvec{a}}
\newcommand{\vA}{\bvec{A}}
\newcommand{\vb}{\bvec{b}}
\newcommand{\vB}{\bvec{B}}
\newcommand{\vc}{\bvec{c}}
\newcommand{\vd}{\bvec{d}}
\newcommand{\vD}{\bvec{D}}
\newcommand{\ve}{\bvec{e}}
\newcommand{\vf}{\bvec{f}}
\newcommand{\vF}{\bvec{F}}
\newcommand{\vg}{\bvec{g}}
\newcommand{\vG}{\bvec{G}}
\newcommand{\vh}{\bvec{h}}
\newcommand{\vi}{\bvec{i}}
\newcommand{\vj}{\bvec{j}}
\newcommand{\vk}{\bvec{k}}
\newcommand{\vK}{\bvec{K}}
\newcommand{\vl}{\bvec{l}}
\newcommand{\vell}{\bvec{\ell}}
\newcommand{\vL}{\bvec{L}}
\newcommand{\vm}{\bvec{m}}
\newcommand{\vp}{\bvec{p}}
\newcommand{\vq}{\bvec{q}}
\newcommand{\vr}{\bvec{r}}
\newcommand{\vs}{\bvec{s}}
\newcommand{\vS}{\bvec{S}}
\newcommand{\vt}{\bvec{t}}
\newcommand{\vT}{\bvec{T}}
\newcommand{\vu}{\bvec{u}}
\newcommand{\vU}{\bvec{U}}
\newcommand{\vv}{\bvec{v}}
\newcommand{\vV}{\bvec{V}}
\newcommand{\vw}{\bvec{w}}
\newcommand{\vW}{\bvec{W}}
\newcommand{\vx}{\bvec{x}}
\newcommand{\vX}{\bvec{X}}
\newcommand{\vy}{\bvec{y}}
\newcommand{\vY}{\bvec{Y}}
\newcommand{\vz}{\bvec{z}}
\newcommand{\vZ}{\bvec{Z}}

\newcommand{\ai}{\avec{\imath}}
\newcommand{\ak}{\avec{k}}
\newcommand{\avi}{\avec{\bvec{\imath}}}
\newcommand{\at}{\avec{t}}
\newcommand{\avt}{\avec{\vt}}
\newcommand{\ax}{\avec{x}}
\newcommand{\ah}{\avec{h}}
\newcommand{\akappa}{\avec{\kappa}}
\newcommand{\avx}{\avec{\vx}}
\newcommand{\ay}{\avec{y}}
\newcommand{\avy}{\avec{\vy}}
\newcommand{\avz}{\avec{\vz}}
\newcommand{\avzero}{\avec{\vzero}}
\newcommand{\aomega}{\avec{\omega}}
\newcommand{\avomega}{\avec{\vomega}}
\newcommand{\anu}{\avec{\nu}}
\newcommand{\avnu}{\avec{\vnu}}
\newcommand{\aDelta}{\avec{\Delta}}
\newcommand{\avDelta}{\avec{\vDelta}}

\newcommand{\valpha}{\bvec{\alpha}}
\newcommand{\vbeta}{\bvec{\beta}}
\newcommand{\vgamma}{\bvec{\gamma}}
\newcommand{\vGamma}{\bvec{\Gamma}}
\newcommand{\vdelta}{\bvec{\delta}}
\newcommand{\vDelta}{\bvec{\Delta}}
\newcommand{\vphi}{\bvec{\phi}}
\newcommand{\vvphi}{\bvec{\varphi}}
\newcommand{\vomega}{\bvec{\omega}}
\newcommand{\vlambda}{\bvec{\lambda}}
\newcommand{\vmu}{\bvec{\mu}}
\newcommand{\vnu}{\bvec{\nu}}
\newcommand{\vpsi}{\bvec{\psi}}
\newcommand{\vepsilon}{\bvec{\epsilon}}
\newcommand{\veps}{\bvec{\varepsilon}}
\newcommand{\veta}{\bvec{\eta}}
\newcommand{\vxi}{\bvec{\xi}}
\newcommand{\vtheta}{\bvec{\theta}}
\newcommand{\vtau}{\bvec{\tau}}
\newcommand{\vzeta}{\bvec{\zeta}}

\newcommand{\hA}{\widehat{A}}
\newcommand{\hvb}{\hat{\vb}}
\newcommand{\hcc}{\widehat{\cc}}
\newcommand{\hD}{\widehat{D}}
\newcommand{\hE}{\widehat{E}}
\newcommand{\hf}{\widehat{f}}
\newcommand{\hF}{\widehat{F}}
\newcommand{\hg}{\hat{g}}
\newcommand{\hvf}{\widehat{\bvec{f}}}
\newcommand{\hh}{\hat{h}}
\newcommand{\hH}{\widehat{H}}
\newcommand{\hi}{\hat{\imath}}
\newcommand{\hI}{\hat{I}}
\newcommand{\hci}{\widehat{\ci}}
\newcommand{\hj}{\hat{\jmath}}
\newcommand{\hp}{\hat{p}}
\newcommand{\hP}{\widehat{P}}
\newcommand{\hS}{\widehat{S}}
\newcommand{\hv}{\hat{v}}
\newcommand{\hV}{\widehat{V}}
\newcommand{\hx}{\hat{x}}
\newcommand{\hX}{\widehat{X}}
\newcommand{\hvX}{\widehat{\vX}}
\newcommand{\hy}{\hat{y}}
\newcommand{\hvy}{\hat{\vy}}
\newcommand{\hY}{\widehat{Y}}
\newcommand{\hvY}{\widehat{\vY}}
\newcommand{\hZ}{\widehat{Z}}
\newcommand{\hvZ}{\widehat{\vZ}}

\newcommand{\halpha}{\hat{\alpha}}
\newcommand{\hvalpha}{\hat{\valpha}}
\newcommand{\hbeta}{\hat{\beta}}
\newcommand{\hvbeta}{\hat{\vbeta}}
\newcommand{\hgamma}{\hat{\gamma}}
\newcommand{\hvgamma}{\hat{\vgamma}}
\newcommand{\hdelta}{\hat{\delta}}
\newcommand{\hvareps}{\hat{\varepsilon}}
\newcommand{\hveps}{\hat{\veps}}
\newcommand{\hmu}{\hat{\mu}}
\newcommand{\hnu}{\hat{\nu}}
\newcommand{\hvnu}{\widehat{\vnu}}
\newcommand{\homega}{\widehat{\omega}}
\newcommand{\hPi}{\widehat{\Pi}}
\newcommand{\hrho}{\hat{\rho}}
\newcommand{\hsigma}{\hat{\sigma}}
\newcommand{\htheta}{\hat{\theta}}
\newcommand{\hTheta}{\hat{\Theta}}
\newcommand{\htau}{\hat{\tau}}
\newcommand{\hxi}{\hat{\xi}}
\newcommand{\hvxi}{\hat{\vxi}}

\newcommand{\otau}{\overline{\tau}}
\newcommand{\oY}{\overline{Y}}

\newcommand{\rD}{\mathring{D}}
\newcommand{\rf}{\mathring{f}}
\newcommand{\rV}{\mathring{V}}

\newcommand{\ta}{\tilde{a}}
\newcommand{\tA}{\tilde{A}}
\newcommand{\tmA}{\widetilde{\mA}}
\newcommand{\tvb}{\tilde{\vb}}
\newcommand{\tcb}{\widetilde{\cb}}
\newcommand{\tB}{\widetilde{B}}
\newcommand{\tc}{\tilde{c}}
\newcommand{\tvc}{\tilde{\vc}}
\newcommand{\tfc}{\tilde{\fc}}
\newcommand{\tC}{\widetilde{C}}
\newcommand{\tcc}{\widetilde{\cc}}
\newcommand{\tD}{\widetilde{D}}
\newcommand{\te}{\tilde{e}}
\newcommand{\tE}{\widetilde{E}}
\newcommand{\tf}{\widetilde{f}}
\newcommand{\tF}{\widetilde{F}}
\newcommand{\tvf}{\tilde{\vf}}
\newcommand{\tcf}{\widetilde{\cf}}
\newcommand{\tg}{\tilde{g}}
\newcommand{\tG}{\widetilde{G}}
\newcommand{\tildeh}{\tilde{h}}
\newcommand{\tH}{\widetilde{H}}
\newcommand{\tch}{\widetilde{\ch}}
\newcommand{\tK}{\widetilde{K}}
\newcommand{\tvk}{\tilde{\vk}}
\newcommand{\tM}{\widetilde{M}}
\newcommand{\tn}{\tilde{n}}
\newcommand{\tN}{\widetilde{N}}
\newcommand{\tQ}{\widetilde{Q}}
\newcommand{\tR}{\widetilde{R}}
\newcommand{\tS}{\widetilde{S}}
\newcommand{\tvS}{\widetilde{\vS}}
\newcommand{\tT}{\widetilde{T}}
\newcommand{\tv}{\tilde{v}}
\newcommand{\tV}{\widetilde{V}}
\newcommand{\tvx}{\tilde{\vx}}
\newcommand{\tW}{\widetilde{W}}
\newcommand{\tx}{\tilde{x}}
\newcommand{\tX}{\widetilde{X}}
\newcommand{\tvX}{\widetilde{\vX}}
\newcommand{\ty}{\tilde{y}}
\newcommand{\tvy}{\tilde{\vy}}
\newcommand{\tz}{\tilde{z}}
\newcommand{\tZ}{\widetilde{Z}}
\newcommand{\tL}{\widetilde{L}}
\newcommand{\tP}{\widetilde{P}}
\newcommand{\tY}{\widetilde{Y}}
\newcommand{\tmH}{\widetilde{\mH}}
\newcommand{\tmK}{\widetilde{\mK}}
\newcommand{\tmM}{\widetilde{\mM}}
\newcommand{\tmQ}{\widetilde{\mQ}}
\newcommand{\tct}{\widetilde{\ct}}
\newcommand{\talpha}{\tilde{\alpha}}
\newcommand{\tdelta}{\tilde{\delta}}
\newcommand{\tDelta}{\tilde{\Delta}}
\newcommand{\tvareps}{\tilde{\varepsilon}}
\newcommand{\tveps}{\tilde{\veps}}
\newcommand{\tlambda}{\tilde{\lambda}}
\newcommand{\tmu}{\tilde{\mu}}
\newcommand{\tnu}{\tilde{\nu}}
\newcommand{\trho}{\tilde{\rho}}
\newcommand{\tvarrho}{\tilde{\varrho}}
\newcommand{\ttheta}{\tilde{\theta}}
\newcommand{\tsigma}{\tilde{\sigma}}
\newcommand{\tvmu}{\tilde{\vmu}}
\newcommand{\tphi}{\tilde{\phi}}
\newcommand{\tPhi}{\widetilde{\Phi}}
\newcommand{\tvphi}{\tilde{\vphi}}
\newcommand{\ttau}{\tilde{\tau}}
\newcommand{\txi}{\tilde{\xi}}
\newcommand{\tvxi}{\tilde{\vxi}}


\newcommand{\mA}{\mathsf{A}}
\newcommand{\mB}{\mathsf{B}}
\newcommand{\mC}{\mathsf{C}}
\newcommand{\vmC}{\bvec{\mC}}
\newcommand{\mD}{\mathsf{D}}
\newcommand{\mF}{\mathsf{F}}
\newcommand{\mG}{\mathsf{G}}
\newcommand{\mH}{\mathsf{H}}
\newcommand{\mI}{\mathsf{I}}
\newcommand{\mK}{\mathsf{K}}
\newcommand{\mL}{\mathsf{L}}
\newcommand{\mM}{\mathsf{M}}
\newcommand{\mP}{\mathsf{P}}
\newcommand{\mQ}{\mathsf{Q}}
\newcommand{\mR}{\mathsf{R}}
\newcommand{\mS}{\mathsf{S}}
\newcommand{\mT}{\mathsf{T}}
\newcommand{\mU}{\mathsf{U}}
\newcommand{\mV}{\mathsf{V}}
\newcommand{\mW}{\mathsf{W}}
\newcommand{\mX}{\mathsf{X}}
\newcommand{\mLambda}{\UpSfLambda}
\newcommand{\mSigma}{\UpSfSigma}
\newcommand{\mzero}{\mathsf{0}}
\newcommand{\mGamma}{\mathsf{\Gamma}}

\newcommand{\bbE}{\mathbb{E}}
\newcommand{\bbF}{\mathbb{F}}
\newcommand{\bbK}{\mathbb{K}}
\newcommand{\bbV}{\mathbb{V}}
\newcommand{\bbZ}{\mathbb{Z}}
\newcommand{\bbone}{\mathbbm{1}}
\newcommand{\naturals}{\mathbb{N}}
\newcommand{\reals}{\mathbb{R}}
\newcommand{\integers}{\mathbb{Z}}
\newcommand{\natzero}{\mathbb{N}_{0}}
\newcommand{\rationals}{\mathbb{Q}}
\newcommand{\complex}{\mathbb{C}}

\newcommand{\ca}{\mathcal{A}}
\newcommand{\cb}{\mathcal{B}}
\providecommand{\cc}{\mathcal{C}}
\newcommand{\cd}{\mathcal{D}}
\newcommand{\cf}{\mathcal{F}}
\newcommand{\cg}{\mathcal{G}}
\newcommand{\ch}{\mathcal{H}}
\newcommand{\ci}{\mathcal{I}}
\newcommand{\cj}{\mathcal{J}}
\newcommand{\ck}{\mathcal{K}}
\newcommand{\cl}{\mathcal{L}}
\newcommand{\cm}{\mathcal{M}}
\newcommand{\tcm}{\widetilde{\cm}}
\newcommand{\cn}{\mathcal{N}}
\newcommand{\cp}{\mathcal{P}}
\newcommand{\calr}{\mathcal{R}}
\newcommand{\cs}{\mathcal{S}}
\newcommand{\ct}{\mathcal{T}}
\newcommand{\cu}{\mathcal{U}}
\newcommand{\cv}{\mathcal{V}}
\newcommand{\cw}{\mathcal{W}}
\newcommand{\cx}{\mathcal{X}}
\newcommand{\tcx}{\widetilde{\cx}}
\newcommand{\cy}{\mathcal{Y}}
\newcommand{\cz}{\mathcal{Z}}

\newcommand{\fc}{\mathfrak{c}}
\newcommand{\fC}{\mathfrak{C}}
\newcommand{\fh}{\mathfrak{h}}
\newcommand{\fu}{\mathfrak{u}}

\newcommand{\me}{\ensuremath{\mathrm{e}}} % for math number 'e', 2.718 281 8..., tha base of natural logarithms
\newcommand{\mi}{\ensuremath{\mathrm{i}}} % for math number 'i', the imaginary unit
\newcommand{\mpi}{\ensuremath{\mathrm{\pi}}} % for math number 'pi', the circumference of a circle of diameter 1




\setlength{\textheight}{24cm} \setlength{\textwidth}{16cm}
\setlength{\hoffset}{-1.3cm} \setlength{\voffset}{-1.8cm}

\newcommand{\supp}{\operatorname{supp}}

\newcommand{\tpmod}[1]{{\;(\operatorname{mod}\;#1)}}

\DeclareMathOperator*{\esssup}{ess\,sup}

%\DeclareMathOperator{\comp}{comp}
\DeclareMathOperator{\SOL}{SOL}
\DeclareMathOperator{\APP}{APP}

\allowdisplaybreaks

\newcommand{\fred}[1]{\begingroup\color{blue}Fred: #1\endgroup}
\newcommand{\peter}[1]{\begingroup\color{purple}#1\endgroup}
\newcommand{\kachi}[1]{\begingroup\color{ForestGreen}#1\endgroup}

\newcommand{\vinfty}{\boldsymbol{\infty}}


\begin{document}

\newtheorem{theorem}{Theorem}
\theoremstyle{definition}
\newtheorem{definition}{Definition}


\title{Generalized Tractability}
\author{\fred{Fred J. Hickernell}, \peter{Peter Kritzer}, \kachi{Onyekachi Osisiogu}, and \\ Henryk Wo\'zniakwoski}
\date{\today}

\maketitle

\section{Introduction}

TODO: We need to check the relation of our work to that of Michael Gnewuch and Henryk. We do not need tensor product structure: do they need that assumption? The summation conditions on the eigenvalues might be a significant difference between our results and theirs.

\medskip

TODO: Check in how far our conditions coincide with existing conditions for less general tractability notions (polynomial tractability, etc.).

\medskip

Let $\cf_d$ and $\cg_d$ be Hilbert spaces, and $\SOL_d : \cf_d \to \cg_d$ be a linear solution operator with adjoint $\SOL_d^*$ such that $\SOL_d^*\SOL_d : \cf_d \to \cf_d$ has eigenvalues and orthonormal eigenvectors
\[
\lambda_{\max}^2 \ge \lambda_{1,d}^2 \ge \lambda_{2,d}^2 \ge \cdots, \qquad u_{1,d}, u_{2,d}, \ldots,
\]
This means that the non-negative $\lambda_{i,d}$ are the singular values of the $\SOL_d$. To avoid trivial cases, we assume that there are an infinite number of positive eigenvalues.


The goal is to find an approximate solution, $\APP_d:\cb_d \times (0,\infty) \to \cg_d$ satisfying
\[
\norm[\cg_d]{\SOL_d(f) - \APP_d(f,\varepsilon)} \le \varepsilon,
\]
where $\cb_d$ is the unit ball in $\cf_d$ and $\APP_d(f,\varepsilon)$ is allowed to depend on arbitrary linear functionals.  In this case, the optimal solution is
\[
\APP_d(f,\varepsilon) = \sum_{i=1}^n \SOL(u_i) \ip[\cf_d]{f}{u_i},
\]
and the complexity of our linear problem is
\begin{equation}\label{eq:comp_def}
\comp(\varepsilon, d) = \min \{n \in \natzero : \lambda_{n+1, d} \le \varepsilon\}.
\end{equation}

By definition, the complexity is non-decreasing  as $\varepsilon$ tends to zero.  We also expect it to be non-decreasing as $d$ tends to infinity.  This assumes that the Hilbert spaces are defined in such a way that $\lambda_{j,\cdot}$ is non-decreasing, although this assumption is not needed for our results.  

The question is,``How hard is this problem as $\varepsilon$ tends to zero or as $d$ tends to infinity?''  We want to bound the complexity in terms of a simple function of $\varepsilon$ and $d$ and identify simple conditions on the singular values that are equivalent to this bound holding.

\section{Main Results}

\subsection{Notation and fundamental definitions}
Let $T$ be a tractability function defined as follows,
\[
T :(0,\infty) \times \mathbb{N} \times [0,\infty)^s \rightarrow (0,\infty),
\]
\fred{I think that we do not need to bound $T$ away from $0$.}
A tractability function is a function of simple form that provides an upper bound on the computational complexity of a problem.  Some examples of tractability functions commonly considered are provided in Section \ref{sec:examples}.


A problem is defined as $T$-tractable with parameter $\vp$ iff there exists a positive constant $C_{\vp}$ such that
\begin{equation} \label{eq:tract_def}
	\comp(\varepsilon,d) \le C_{\vp}\, T(\varepsilon^{-1},d,\vp) \qquad \forall \varepsilon >0, \ d \in \mathbb{N}.
\end{equation}
A problem is \emph{strongly}
$T$-tractable with parameter $\vp$ iff the complexity is independent of the dimension of the problem, that is, there exists a positive constant $C_{\vp}$ such that
\begin{equation} \label{eq:strong_tract_def}
	\comp(\varepsilon,d) \le C_{\vp}\, T(\varepsilon^{-1},1,\vp) \qquad \forall \varepsilon >0 , \ d \in \mathbb{N}.
\end{equation}
In this article we allow $\vp$ to be a scalar or vector as the situation dictates.

Note that the tractability function is bounded below away from zero even as $\varepsilon \to \infty$, so that the information based complexity is not expected to vanish under arbitrarily loose error requirements.  For this definition of tractability to make sense we assume that
\begin{subequations} \label{eq:Tconditions}
\begin{equation}
	T \text{ is non-decreasing in all variables}
\end{equation}
which implies that the problem does not become easier by decreasing the tolerance, $\varepsilon$, or increasing the dimension, $d$. Furthermore, increasing $\vp$ does not decrease $T(\cdot, \cdot, \vp)$.  Since there are an infinite number of positive eigenvalues, it also make sense to assume that
\begin{equation}
	\lim_{\varepsilon \to 0} T(\varepsilon^{-1},d,\vp) = \infty \qquad \forall d \in \naturals, \ \vp \in [0,\infty)^s.
\end{equation}
Since $T(\cdot,d,\vp)$ is non-decreasing, we may define the following limit:
\begin{equation}
	T(0,d,\vp):=\lim_{\varepsilon\to\infty}T(\varepsilon^{-1},d,\vp).
\end{equation}
Finally, there is a technical assumption required for our analysis.  There exists a $K_{\vp,\tau}$ depending on $\vp$ and $\tau$, but  independent of $\varepsilon$ and $d$, such that
\begin{equation} \label{eq:ptauassume}
	[T(\varepsilon^{-1},d,\vp)]^\tau \le K_{\vp,\tau} T(\varepsilon^{-1},d,\tau \vp),   \quad \forall \varepsilon \in (0,\infty), \ d \in \naturals, \ \vp\in[0,\infty)^s, \ \tau\in [1,\infty). 
\end{equation}
\end{subequations}

If \eqref{eq:tract_def} holds for some $\vp$, it clearly holds for larger, $\vp$.  We are often interested in the optimal or smallest $\vp$ for which \eqref{eq:tract_def} holds.  Define the closures of the sets of parameters for which our (strong) tractability conditions hold:
\begin{equation*}
	\cp_{\textup{trct}} : = \{\vp^* : \eqref{eq:tract_def} \text{ holds }\forall \vp \in (\vp^*,\boldsymbol{\infty})\}, \qquad
	\cp_{\textup{strct}} : = \{\vp^* : \eqref{eq:strong_tract_def} \text{ holds }\forall \vp \in (\vp^*,\boldsymbol{\infty})\}.
\end{equation*}
The  set of optimal parameters is defined as all of those parameters satisfying the (strong) tractability conditions that are not greater than or equal to others:
\begin{gather}
	\label{eq:Poptdef}
	\mathcal{P}_{\textup{opt}} : = \{\vp^* \in \cp_{\textup{trct}} :  \vp^* \notin [\vp^\dagger,\boldsymbol{\infty}) \ \forall \vp^\dagger \in  \cp_{\textup{trct}} \setminus \{\vp^*\} \}, \\
	\label{eq:Psoptdef}
	\mathcal{P}_{\textup{sopt}} : = \{\vp^* \in \cp_{\textup{strct}} :  \vp^* \notin [\vp^\dagger,\boldsymbol{\infty}) \ \forall \vp^\dagger \in  \cp_{\textup{strct}} \setminus \{\vp^*\} \}.
\end{gather}

Certain common notions of tractability are covered by the general situation described above.
\begin{equation*}
	\begin{array}{r@{\qquad}c@{\qquad}cc}
		& T(\varepsilon^{-1},d,\vp)
		& T(0,d,\vp)\\
		\toprule
		\text{strong polynomial algebraic  tractability} & \max(1,\varepsilon^{-p}) & 1\\
		\text{polynomial algebraic tractability} & \max(1,\varepsilon^{-p})d^{q} & d^{q}\\
		\text{strong polynomial exponential tractability} &  [\max(1,\log(1 + \varepsilon^{-1})]^p & 1\\
		\text{polynomial exponential tractability} &
		[\max(1,\log(1 + \varepsilon^{-1}))]^p  d^{q} & d^{q}
	\end{array}
\end{equation*}
Here above, $\vp = (p,q)$ in some cases.


In the sections below we prove necessary and sufficient conditions on tractability as generally defined in \eqref{eq:tract_def} and \eqref{eq:strong_tract_def}.  These conditions involve the boundedness of sums defined in terms of $\{T(\lambda_{i,d}^{-1},\cdot, \cdot)\}_{i=1}^\infty$.  In practice, it may be easier to verify whether or not these conditions hold than to verify \eqref{eq:comp_def} directly.


\subsection{Strong tractability}

We first consider the simpler case when the complexity is essentially independent of the dimension, $d$.  The proof also introduces the line of argument used for the case where there is $d$ dependence.

\begin{theorem}\label{thm_main_strong_tract2}
A problem is strongly $T$-tractable iff there exists $\vp \in [\vzero, \boldsymbol{\infty})$ and an integer $L_{\vp} > 0$ such that
\begin{equation} \label{eq:strong_tractiff3}
     S_{\vp}:=\sup_{d \in \naturals} \sum_{i = L_{\vp}}^\infty \frac{1}{T(\lambda_{i,d}^{-1},1,\vp)} < \infty.
\end{equation}
If \eqref{eq:strong_tractiff3} holds for some $\vp$, let  $\widetilde{\cp}_{\textup{strct}} : = \{\vp^* : \eqref{eq:strong_tractiff3} \text{ holds }\forall \vp \in (\vp^*,\boldsymbol{\infty})\}$.  Then $\cp_{\textup{strct}} = \widetilde{\cp}_{\textup{strct}}$, and the set of optimal strong tractability parameters is
\[
	\cp_{\textup{sopt}} =
	\{\vp^* \in \widetilde{\cp}_{\textup{strct}} :  \vp* \notin [\vp^\dagger,\boldsymbol{\infty}) \ \forall \vp^\dagger \in  \widetilde{\cp}_{\textup{strct}} \setminus \{\vp^*\} \}.
\]
\end{theorem}


\begin{proof}
\textbf{Sufficient condition:}\newline
We make the first part of the argument in some generality so that it can be reused in the proof of Theorem \ref{thm_main_tract2} for tractability.

Since the $\lambda_{i,d}$ are non-increasing in $i$, it follows that the $T(\lambda_{i,d}^{-1},d',\vp)$ are non-decreasing in $i$. In particular, for all $n\in \mathbb{N}$, we have
\begin{align*}
    \lambda_{n+1,d} \le \lambda_{n,d} \le \cdots \le \lambda_{1,d}
    & \implies \frac{1}{T(\lambda_{n+1, d}^{-1},d',\vp)} \le \frac{1}{T(\lambda_{n, d}^{-1},d',\vp)} \le \cdots \le \frac{1}{T(\lambda_{1, d}^{-1},d',\vp)} \\
    & \implies \frac{1}{T(\lambda_{n+1, d}^{-1},d',\vp) }
    \le \frac 1n \sum_{i=1}^n  \frac{1}{T(\lambda_{i, d}^{-1},d',\vp) }
    \le \frac 1n \sum_{i=1}^\infty  \frac{1}{T(\lambda_{i, d}^{-1},d',\vp)}.
\end{align*}
Thus, we can conclude from the previous line that
\begin{multline} \label{eq:implicate_a}
    n \ge T(\varepsilon^{-1},d',\vp) \sum_{i=1}^\infty \frac{1}{T(\lambda_{i, d}^{-1},d',\vp)} \\
   \implies   \frac{1}{T(\lambda_{n+1, d}^{-1},d',\vp)} \le
   \frac 1n \sum_{i=1}^\infty \frac{1}{T(\lambda_{i, d}^{-1},d',\vp) } \le \frac{1}{T(\varepsilon^{-1},d',\vp)} \qquad \forall \varepsilon > 0, \ d,d' \in \naturals.
\end{multline}

Moreover, given that $T$ is increasing in its arguments, we have the following equivalent expression for the complexity via \eqref{eq:comp_def}:
    \begin{equation*} %\label{eq:comp_equiv}
    \comp(\varepsilon, d) = \min \{n \in \natzero : \lambda_{n+1, d} \le \varepsilon\} = \min \biggl\{n \in \natzero : \frac{1}{T(\lambda_{n+1,d}^{-1},d',\vp)}\le \frac{1}{T(\varepsilon^{-1},d',\vp) }\biggr\},
\end{equation*}
for any $d' \in \naturals$.  Combining this expression for the complexity with the implication in \eqref{eq:implicate_a} implies an upper bound on the complexity: \fred{think $1+$ not needed}
\begin{equation} \label{eq:compUP}
	\comp(\varepsilon,d)
	\le T(\varepsilon^{-1},d',\vp) \sum_{i=1}^\infty \frac{1}{T(\lambda_{i, d}^{-1},d',\vp)} \qquad \forall \varepsilon > 0, \ d, d' \in \naturals.
\end{equation}

Now we take this upper bound further, specializing to the case of $d'=1$:
\begin{align*}
       \comp(\varepsilon,d)
       & = T(\varepsilon^{-1},1, \vp) \Biggl [ \sum_{i=1}^{L_{\vp}-1} \frac{1}{T(\lambda_{i, d}^{-1},1,\vp)}
       + \underbrace{\sum_{i=L_{\vp}}^\infty \frac{1}{T(\lambda_{i, d}^{-1},1,\vp)}}_{\le S_{\vp} <\infty\ \text{by \eqref{eq:strong_tractiff3}} }\Biggr] \\
       & \le T(\varepsilon^{-1},1,\vp) \underbrace{\left[\frac{(L(p)-1)}{T(\lambda_{\max}^{-1},1,\vp)} + S_{\vp}\right]}_{=:C_{\vp}}\\
       & =  C_{\vp}T(\varepsilon^{-1},1,\vp).
%       & \le \underbrace{2 C_{1,\vp}}_{\text{since } T(\varepsilon^{-1},1,\vp) \ge 1} T(\varepsilon^{-1},1,\vp)
%       = \underbrace{C_{\vp}}_{:=2 C_{1,\vp}}T(\varepsilon^{-1},1,\vp).
\end{align*}
This means that we have strong $T$-tractability via \eqref{eq:strong_tract_def}, and verifies the sufficiency of \eqref{eq:strong_tractiff3}.



\bigskip
\noindent \textbf{Necessary condition:} \\
Suppose that we have strong
$T$-tractability as defined in \eqref{eq:strong_tract_def}. That is, for some $\vp \ge \vzero$ there exists a positive constant $C_{\vp}$ such that
\[
\comp(\varepsilon,d)\le C_{\vp}\, T(\varepsilon^{-1},1,\vp)
\qquad \forall \varepsilon > 0, \ d \in \mathbb{N}.
\]
Since the sequence of eigenvalues $\lambda_{1,d}, \lambda_{2,d}, \ldots $ is non-increasing, we have
\begin{equation}\label{eq:lambda_K_strong3}
\lambda_{\lfloor C_{\vp}\, T(\varepsilon^{-1},1,\vp)\rfloor +1,d}\le \varepsilon.
\end{equation}

Define
\[
j (\varepsilon,\vp):= \lfloor C_{\vp}\, T(\varepsilon^{-1},1,\vp)\rfloor +1, \quad
j (\infty,\vp) = \lim_{\varepsilon \to \infty} j (\varepsilon,\vp) \ge 1 .
\]
Thus, it follows by \eqref{eq:lambda_K_strong3} that $\lambda_{j(\varepsilon,\vp),d} \le \varepsilon$.
Note furthermore that we always have
\[
j(\varepsilon,\vp)\le C_{\vp}\, T(\varepsilon^{-1},1,\vp)+1 \le C_{\vp} T(\lambda_{j(\varepsilon,\vp),d}^{-1},1,\vp)+1, \qquad \forall \varepsilon > 0,
\]
since
$T(\cdot,1, \vp)$ is non-decreasing.

For $\varepsilon$ taking on all positive values, $j(\varepsilon,\vp)$ takes on all values greater than or equal to $j(\infty,\vp)$, so
\[
j\le C_{\vp} T(\lambda_{j,d}^{-1},1,\vp)+1, \qquad \forall j \ge j(\infty,\vp).
\]
This implies via our technical assumption \eqref{eq:ptauassume} that
\begin{gather*}
 K_{\vp,\tau}\,T (\lambda_{j,d}^{-1},1,\tau \vp) \ge
 [T(\lambda_{j,d}^{-1},1, \vp)]^\tau
 \ge
  \left[\frac{(j-1)}{C_{\vp}}\right]^\tau \qquad \forall j \ge j(\infty,\vp) \\
 \frac{1}{T (\lambda_{j,d}^{-1},1,\tau \vp)} \le
\frac{K_{\vp,\tau}\, C_{\vp}^\tau}{(j-1)^\tau}\qquad \forall j \ge j(\infty,\vp)
\end{gather*}
Summing both sides of this last inequality for $j$ from $j(\infty,\vp) \ge 1$ to $\vinfty$ yields 
\begin{align*}
\sup_{d\in\naturals} \sum_{j=j (\infty,\vp)}^\infty \frac{1}{T (\lambda_{j,d}^{-1},1, \tau \vp)}
& \le  \sup_{d\in\naturals} \frac{1}{T (\lambda_{1,d}^{-1},1, \tau \vp)} + K_{\vp,\tau}\, C_{\vp}^\tau
\sum_{j=2}^\infty \frac{1}{(j-1)^\tau} \\
& \le\frac{1}{T (\lambda_{\max}^{-1},1, \tau \vp)} +  K_{\vp,\tau}\, C_{\vp}^\tau
\zeta (\tau)  < \infty,
\end{align*}
where $\zeta$ denotes the Riemann zeta function.
This yields \eqref{eq:strong_tractiff3} with $\vp$ replaced by $\vp' = \tau \vp$ and $L_{\vp'} = j (\infty,\vp)$, and so we see the necessity of $\eqref{eq:strong_tractiff3}$.

\bigskip
\noindent \textbf{Optimality:} \\
To complete the proof, we must show that $\cp_{\textup{strct}} = \widetilde{\cp}_{\textup{strct}}$.  Then, the expression for  $\cp_{\textup{sopt}}$ automatically follows from its definition in \eqref{eq:Psoptdef}.

First we show  that if $\vp^* \in \cp_{\textup{strct}}$, then $\vp^* \in \widetilde{\cp}_{\textup{strct}}$.  If $\vp^* \in \cp_{\textup{strct}}$, then \eqref{eq:strong_tract_def} must hold for all $\vp' \in (\vp^*,\vinfty)$.  For any $\vp' \in (\vp^*,\vinfty)$, we may choose $\vp$ and $\tau$ such that $\vp^* < \vp < \vp' = \tau \vp$. Since $\vp > \vp^*$, \eqref{eq:strong_tract_def} also holds for $\vp$, and it follows that \eqref{eq:strong_tractiff3} must hold for $\vp'$ by the proof of necessity above.  Thus, $\vp^* \in \widetilde{\cp}_{\textup{strct}}$.

Next we show that if $\vp^* \in \widetilde{\cp}_{\textup{strct}}$, then $\vp^* \in \cp_{\textup{strct}}$.  If $\vp^* \in \widetilde{\cp}_{\textup{strct}}$, then \eqref{eq:strong_tractiff3}  holds for all $\vp \in (\vp^*,\vinfty)$.  By the argument to prove the sufficient condition above, it follows that \eqref{eq:strong_tract_def} must also hold for all $\vp \in (\vp^*,\vinfty)$. Thus, $\vp^* \in \cp_{\textup{strct}}$.

\bigskip

\noindent This concludes the proof of the theorem.

\end{proof}



\subsection{Tractability} \label{sec:tractability}


The argument proving equivalent conditions for tractability is similar to, but somewhat more involved than the proof of Theorem \ref{thm_main_strong_tract2}.  Moreover, the specification of the set of optimal parameters is not as precise as in Theorem \ref{thm_main_strong_tract2}.

\begin{theorem}\label{thm_main_tract2}
A problem is $T$-tractable iff there exist $\vp\ge \vzero$ and a constant $L_{\vp} > 0$ such that
\begin{equation} \label{eq:tractiff4}
     S_{\vp}:=\sup_{d \in \naturals}
     \sum_{i = \lceil L_{\vp}\, T(0,d,\vp) \rceil}^\infty \frac{1}{T(\lambda^{-1}_{i,d},d,\vp)}< \infty.
\end{equation}
If \eqref{eq:tractiff4} holds for some $\vp$, let  $\widetilde{\cp}_{\textup{trct}} : = \{\vp^* : \eqref{eq:tractiff4} \text{ holds }\forall \vp \in (\vp^*,\boldsymbol{\infty})\}$, and   $\widehat{\cp}_{\textup{trct}} : = \{\vp^* : 2\vp \in \widetilde{\cp}_{\textup{trct}}\}$.  Then $\cp_{\textup{trct}} = \widetilde{\cp}_{\textup{trct}}$, and the set of optimal strong tractability parameters is
\[
\cp_{\textup{opt}} =
\{\vp^* \in \widetilde{\cp}_{\textup{trct}} :  \vp^* \notin [\vp^\dagger,\boldsymbol{\infty}) \ \forall \vp^\dagger \in  \widetilde{\cp}_{\textup{trct}} \setminus \{\vp^*\} \}.
\]\end{theorem}


\begin{proof}
    \textbf{Sufficient Condition:}\\
Suppose that \eqref{eq:tractiff4} holds for some $\vp \ge 0$. By the argument in the proof of the sufficient condition for Theorem \ref{thm_main_strong_tract2}, it follows from \eqref{eq:compUP} with $d'=d$ that the complexity has an upper bound of
\begin{equation*}
	 \comp(\varepsilon,d) \le  T(\varepsilon^{-1},d,\vp) \sum_{i=1}^\infty \frac{1}{T(\lambda_{i, d}^{-1},d,\vp)}
\end{equation*}

Let $J(\vp,d): = \lceil L_{\vp}\, T(0,d,\vp)\rceil$.  Then this upper bound on the complexity can be further bounded above by
\begin{align*}
       \comp(\varepsilon,d)
       & = T(\varepsilon^{-1},d,\vp) \Biggl [ \sum_{i=1}^{J(\vp,d) -1} \frac{1}{T(\lambda_{i, d}^{-1},d,\vp)}
       + \underbrace{\sum_{i=J(\vp,d)}^\infty \frac{1}{T(\lambda_{i, d}^{-1},d,\vp)}}_{\le S_{\vp} \text{ by \eqref{eq:tractiff4}}} \Biggr ] \\
       & \le T(\varepsilon^{-1},d,\vp)\left[ \frac{J(\vp,d)-1}{T(\lambda_{\max}^{-1},d,\vp)} + S_{\vp}\right]
        \\
       & \le T(\varepsilon^{-1},d,\vp)\left[\frac{L_{\vp}\, T(0,d,\vp)}{T(\lambda_{\max}^{-1},1,\vp)} + S_{\vp}\right] \\
       & \le\frac{L_{\vp}}{T(\lambda_{\max}^{-1},1,\vp)} \,T(\varepsilon^{-1},d,\vp) T(0,d,\vp) + S_{\vp}\, T(\varepsilon^{-1},d,\vp)\\
       & \le\frac{L_{\vp}}{T(\lambda_{\max}^{-1},1,\vp)} \,T(\varepsilon^{-1},d,\vp) T(\varepsilon^{-1},d,\vp) + S_{\vp}\, T(\varepsilon^{-1},d,\vp)\\
       & \le \frac{L_{\vp}\,K_{\vp,2}}{T(\lambda_{\max}^{-1},1,\vp)}  T(\varepsilon^{-1},d,2\vp)  + S_{\vp}\, T(\varepsilon^{-1},d,\vp)  \qquad \text{by \eqref{eq:ptauassume}}\\
       & \le \underbrace{\left[\frac{L_{\vp}\,K_{\vp,2}}{T(\lambda_{\max}^{-1},1,\vp)} + S_{\vp}\right]}_{=:C_{2\vp}} \,T(\varepsilon^{-1},d,2\vp)
\end{align*}
It follows that we have $T$-tractability, which shows sufficiency of \eqref{eq:tractiff4}. \\



\noindent \textbf{Necessary condition:}\\
Suppose that we have
$T$-tractability. That is, for some $\vp \ge \vzero$, there exists a positive constant $C_{\vp}$ such that
\[
\comp(\varepsilon,d)\le C_{\vp}\, T(\varepsilon^{-1},d,\vp) \qquad \forall \varepsilon > 0,  \ d \in \naturals.
\]
Since the sequence of eigenvalues $\lambda_{1,d}, \lambda_{2,d}, \ldots $ is non-increasing, we have
\begin{equation}\label{eq:lambda_K4}
	\lambda_{\lfloor C_{\vp}\, T(\varepsilon^{-1},d,\vp)\rfloor +1,d}\le \varepsilon \qquad \forall \varepsilon > 0,  \ d \in \naturals..
\end{equation}
Define the integers
\[
j (\varepsilon,d,\vp):= \lfloor C_{\vp}\, T(\varepsilon^{-1},d,\vp)\rfloor +1, \quad
j (\infty,d,\vp) = \lfloor C_{\vp}\, T(0,d,\vp)\rfloor +1.
\]
Thus, it follows by \eqref{eq:lambda_K4} that $\lambda_{j(\varepsilon,d,\vp),d} \le \varepsilon$.
Note furthermore that we always have
\[
j(\varepsilon,d,\vp)\le C_{\vp}\, T(\varepsilon^{-1},d,\vp)+1 \le C_{\vp} T(\lambda_{j(\varepsilon,d,\vp),d}^{-1},d,\vp)+1, \qquad \forall \varepsilon > 0, \ d \in \naturals,
\]
since
$T(\cdot,d,\vp)$ is non-decreasing.

For $\varepsilon$ taking on all positive values, $j(\varepsilon,d,\vp)$ takes on all values greater than or equal to $j(\infty,d,\vp)$, so
\[
j\le C_{\vp}\, T(\varepsilon^{-1},d,\vp)+1 \le C_{\vp} T(\lambda_{j,d}^{-1},d,\vp)+1, \qquad \forall j \ge j(\infty,d,\vp).
\]
This implies via our technical assumption \eqref{eq:ptauassume} that
\begin{equation*}
	K_{\vp,\tau}\,T (\lambda_{j,d}^{-1},d,\tau \vp) \ge
	[T(\lambda_{j,d}^{-1},d, \vp)]^\tau
	\ge
	\left[\frac{(j-1)}{C_{\vp}}\right]^\tau \qquad \forall j \ge j(\infty,d,\vp)
\end{equation*}
Summing both sides of the inequality yields and noting that $L_{\vp} \le L(\tau\vp)$, it follows that
\peter{(Peter: Why does $L(\cdot)$ ocur here?)} \fred{Because I got lost a bit.}

\begin{multline*}
	\sup_{d\in\naturals} \sum_{j=j(\infty,d,\vp)}^\infty \frac{1}{T (\lambda_{j,d}^{-1},d, \tau \vp)} \le
	\sup_{d\in\naturals} \sum_{j=j(\infty,d,\vp)}^\infty \frac{1}{T (\lambda_{j,d}^{-1},d, \tau \vp)} \\
	\le K_{\vp,\tau}\, C_{\vp}^\tau
	\sum_{j=j(\infty,d,\vp)}^\infty \frac{1}{(j-1)^\tau} \le
	K_{\vp,\tau}\, C_{\vp}^\tau\, \zeta (\tau) < \infty,
\end{multline*}
where $\zeta$ denotes the Riemann zeta function.


Note now that, \peter{for any $d\in\naturals$,}
\[
 j(\infty,d,\vp) = \lfloor C_{\vp}\, T(0,d,\vp)\rfloor +1 \le \lfloor C_{\vp}^\tau\, T(0,d,\tau \vp)\rfloor +1.
\]
So, there exists $L(\tau \vp)$ such that
\[
\peter{\sup_{d\in\naturals}}\,\, \sum_{j=\lceil L(\tau \vp) T(0,d,\tau \vp)\rceil}^\infty \frac{1}{T(\lambda_{j,d}^{-1},d, \tau\vp)}<\infty.
\]
\fred{Fred:  Not sure how $C_{\vp}^\tau$ becomes $L(\tau \vp)$?}

\peter{Peter: You are right. We could probably solve this by just saying above that
\[
 j(\infty,d,\vp) = \lfloor C_{\vp}\, T(0,d,\vp)\rfloor +1 \le \lfloor C_{\vp}\, T(0,d,\tau \vp)\rfloor +1,
\]
and then introduce $L(\tau \vp)$ to bound $C_{\vp}$ and also incorporate the ``+1''?
}

This yields \eqref{eq:tractiff4} with $\vp$ replaced by $\tau \vp$, and so we see the necessity of $\eqref{eq:tractiff4}$.

\bigskip

\noindent \textbf{Optimality:}
\fred{Not sure how to make optimality works because for sufficiency $\vp$ turns into $2\vp$.}

\end{proof}


\section{Examples} \label{sec:examples}

\paragraph{Example 1: Strong polynomial tractability}

Let $T: (0,\infty) \times \naturals \times [0,\infty) \times [0,\infty)\Rightarrow [1,\infty)$
be defined by
\[
 T(\varepsilon^{-1},d,\vp)= (\max\{\varepsilon^{-1},1\})^p .
\]
Then Theorem \ref{thm_main_strong_tract2} yields the iff-condition
\[
 \sup_{d\in\naturals} \sum_{i=L(p)}^\infty (\min\{\lambda_{i,d},1\})^p < \infty.
\]
This essentially recovers the result on strong polynomial tractability in \cite[Theorem 5.1]{NW08}

\paragraph{Example 2: Polynomial tractability}

Let $T: (0,\infty) \times \naturals \times [0,\infty) \times [0,\infty)\Rightarrow [1,\infty)$
be defined by
\[
 T(\varepsilon^{-1},d,\vp)= (\max\{\varepsilon^{-1},1\})^p\, d^q.
\]
Then Theorem \ref{thm_main_tract2} yields the iff-condition
\[
 \sup_{d\in\naturals} \sum_{i=L_{\vp} d^{q_1}}^\infty (\min\{\lambda_{i,d},1\})^p d^{-q} < \infty.
\]
This essentially recovers the result on polynomial tractability in \cite[Theorem 5.1]{NW08}



\peter{(PETER: Add further examples (e.g., EXP- polynomial tractability, etc.)}

\bigskip

\fred{We need to check the proof for
\begin{itemize}
\item correctness
\item exponents
\item whether the additional assumptions are too restrictive and can be generalized
\item whether we need $\tau$ in the assumption
\item Do we need non-decreasing or strictly increasing?
\end{itemize}}
\begin{definition}[Weak Tractability]
If $n(\varepsilon,d)$ denotes the minimal number of function values needed to approximate all functions. We say that the problem is \emph{weakly tractable} if
    \[\lim_{\varepsilon^{-1}+d\rightarrow \infty} \frac{\ln n(\varepsilon,d)}{\varepsilon^{-1}+d} = 0\]
\end{definition}
\begin{definition}[Quasi-polynomial Tractability]
    The problem is \emph{quasi-polynomially tractable}, if there exist two constants $C, t> 0$ such that
    \[
    n(\varepsilon,d) \leq C\exp\{t(1+\ln(1/\varepsilon))(1+\ln(d))\}
    \] for all $\varepsilon \in (0,1)$ and $d\in\mathbb{N}$.
\end{definition}
\begin{thebibliography}{99}

\bibitem{NW08} E.~Novak,H.~Wo\'zniakowski. \textit{Tractability of Multivariate Problems, Volume I: Linear Information}.
European Mathematical Society Publishing House, Zurich, 2008.
 ×
\end{thebibliography}


\end{document}
