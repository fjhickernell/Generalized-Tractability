\documentclass[11pt,a4paper]{article}
\usepackage{latexsym,amsfonts,amsmath,graphics,mathtools,amssymb,amsthm,booktabs}
\usepackage{epsfig}
\usepackage[notref,notcite]{showkeys}
%\usepackage[active]{srcltx}
\usepackage{enumitem}
\usepackage[usenames,dvipsnames,svgnames,table]{xcolor}

\usepackage{algorithmic}

%\input{common-macros}
%\input{common-macros2}

\input{FJHDef.tex}



\setlength{\textheight}{24cm} \setlength{\textwidth}{16cm}
\setlength{\hoffset}{-1.3cm} \setlength{\voffset}{-1.8cm}

\newcommand{\supp}{\operatorname{supp}}

\newcommand{\tpmod}[1]{{\;(\operatorname{mod}\;#1)}}

\DeclareMathOperator*{\esssup}{ess\,sup}

%\DeclareMathOperator{\comp}{comp}
\DeclareMathOperator{\SOL}{SOL}
\DeclareMathOperator{\APP}{APP}

\allowdisplaybreaks

\newcommand{\fred}[1]{\begingroup\color{blue}#1\endgroup}
\newcommand{\peter}[1]{\begingroup\color{purple}#1\endgroup}
\newcommand{\kachi}[1]{\begingroup\color{ForestGreen}#1\endgroup}


\begin{document}

\newtheorem{theorem}{Theorem}
\theoremstyle{definition}
\newtheorem{definition}{Definition}


\title{Generalized Tractability}
\author{\fred{Fred J. Hickernell}, \peter{Peter Kritzer}, \kachi{Onyekachi Osisiogu}, and \\ Henryk Wo\'zniakwoski}
\date{\today}

\maketitle

\section{Introduction}

TODO: We need to check the relation of our work to that of Michael Gnewuch and Henryk. We do not need tensor product structure: do they need that assumption? The summation conditions on the eigenvalues might be a significant difference between our results and theirs.

\medskip

TODO: Check in how far our conditions coincide with existing conditions for less general tractability notions (polynomial tractability, etc.).

\medskip

Let $\cf_d$ and $\cg_d$ Hilbert spaces, and $\SOL_d : \cf_d \to \cg_d$ be a linear solution operator with adjoint $\SOL_d^*$ such that $\SOL_d^*\SOL_d : \cf_d \to \cf_d$ has eigenvalues and orthonormal eigenvectors  
\[
\lambda_{\max}^2 \ge \lambda_{1,d}^2 \ge \lambda_{2,d}^2 \ge \cdots, \qquad u_{1,d}, u_{2,d}, \ldots, 
\]
This means that the non-negative $\lambda_{i,d}$ are the singular values of the $\SOL_d$. The goal is to find an approximate solution, $\APP_d:\cb_d \times (0,\infty) \to \cg_d$ satisfying 
\[
\norm[\cg_d]{\SOL_d(f) - \APP_d(f,\varepsilon)} \le \varepsilon,
\]
where $\cb_d$ is the unit ball in $\cf_d$ and $\APP_d(f,\varepsilon)$ is allowed to depend on arbitrary linear functionals.  In this case, the optimal solution is 
\[
\APP_d(f,\varepsilon) = \sum_{i=1}^n \SOL(u_i) \ip[\cf_d]{f}{u_i},
\]
and the complexity of our linear problem is
\[
\comp(\varepsilon, d) = \min \{n \in \natzero : \lambda_{n+1, d} \le \varepsilon\}.
\]

Certain kinds of common notions of tractability are 
\begin{equation*}
    \begin{array}{r@{\qquad}c@{\qquad}c}
    & \comp(\varepsilon, d)\\
    \toprule
    \text{strong polynomial algebraic  tractability} & \Theta(\varepsilon^{-p})\\
    \text{polynomial algebraic tractability} & \Theta(\varepsilon^{-p}d^{q})\\
    \text{strong polynomial exponential tractability} &  \Theta\bigl([\log(1 + \varepsilon^{-1})]^p\bigr)\\
    \text{polynomial exponential tractability} & \Theta\bigl([\log(1 + \varepsilon^{-1})]^p  d^{q}\bigr)
    \end{array}
\end{equation*}


\section{Main Results}


Let $T$ be a tractability function defined as follows,
\[
T :(0,\infty) \times \mathbb{N} \times [0,\infty) \times [0,\infty) \rightarrow [1,\infty),
\] 
which is non-decreasing in all variables.

Also, {we assume that, for any choice of $p\in[0,\infty)$, $q\in [0,\infty)$, and $\tau\in (1,\infty)$,
\[
[T(\varepsilon^{-1},d,p,q)]^\tau \le K_{p,q,\tau} T(\varepsilon^{-1},d,\tau p,\tau q),
\]
where $K_{p,q,\tau}$ is a term that may depend on $p$, $q$, and $\tau$, but is independent of $\varepsilon$ and $d$. 

Furthermore, we assume that, for any choice of $d\in\naturals$ and $p,q\in [0,\infty)$, 
the quantity $T(0,d,p,q):=\liminf_{\varepsilon\to\infty}T(\varepsilon^{-1},d,p,q)$ exists.


A problem is $T$-tractable with parameters $p$ and $q$ iff there exists a positive constant $C_{p,q}$ such that
\begin{equation} \label{comp_def}
\comp(\varepsilon,d) \le C_{p,q}\, T(\varepsilon^{-1},d,p,q) \qquad \forall \varepsilon \in (0,\infty), \ \forall d \in \mathbb{N}.
\end{equation}
The  set of optimal parameters, $\mathcal{P}_{\text{opt}}$ is defined as
\begin{multline}
    \mathcal{P}_{\text{opt}} : = \{(p^*,q^*) : \eqref{comp_def} \text{ holds }\forall (p,q) \in [p^*,\infty) \times [q^*,\infty) \setminus (p^*,q^*) \\ 
    \eqref{comp_def} \text{ fails } \forall (p,q) \in [0,p^*] \times [0,q^*] \setminus (p^*,q^*) \}.
\end{multline}




\begin{theorem}\label{thm_main_strong_tract2} 
A problem is $T$-tractable with $q=0$ iff there exists $p>0$ and an integer $L(p) > 0$ such that
\begin{equation} \label{eq:strong_tractiff3}
     S_p:=\sup_{d \in \naturals} \sum_{i = L(p)}^\infty \frac{1}{T(\lambda_{i,d}^{-1},1,p,0)} < \infty.
\end{equation}
If \eqref{eq:strong_tractiff3} holds, then $\mathcal{P}_{\text{opt}}=(p^*,0)$, where
\[
 p^* := \inf\{p>0\ \mbox{such that \eqref{eq:strong_tractiff3} holds}\}.
\]
\end{theorem}
\begin{proof}
\textbf{Sufficient condition:}\\
Given that $T$ is increasing in its arguments, we have the following equivalent expression, 
    \begin{align*}
    \comp(\varepsilon, d) &= \min \{n \in \natzero : \lambda_{n+1, d} \le \varepsilon\} \\
    &= \min \biggl\{n \in \natzero : \frac{1}{T(\lambda_{n+1,d}^{-1},1,p,0)}\le \frac{1}{T(\varepsilon^{-1},1,p,0) }\biggr\}.
\end{align*}
Also since the $\lambda_{i,d}$ are non-increasing, it follows that the $T(\lambda_{i,d}^{-1},1,p,0)$ are non-decreasing. In particular, for all $n\in \mathbb{N}$, we have

\begin{align*}
    \MoveEqLeft{\lambda_{n+1,d} \le \lambda_{n,d} \le \cdots \le \lambda_{1,d}} \\
    & \implies \frac{1}{T(\lambda_{n+1, d}^{-1},1,p,0)} \le \frac{1}{T(\lambda_{n, d}^{-1},1,p,0)} \le \cdots \le \frac{1}{T(\lambda_{1, d}^{-1},1,p,0)} \\
    & \implies \frac{1}{T(\lambda_{n+1, d}^{-1},1,p,0) }
    \le \frac 1n \sum_{i=1}^n  \frac{1}{T(\lambda_{i, d}^{-1},1,p,0) }
    \le \frac 1n \sum_{i=1}^\infty  \frac{1}{T(\lambda_{i, d}^{-1},1,p,0)}.
\end{align*}
Thus, we can conclude that 
\begin{align*}
    n \ge T(\varepsilon^{-1},1,p,0) \sum_{i=1}^\infty \frac{1}{T(\lambda_{i, d}^{-1},1,p,0)}
    & \implies 
\frac 1n \sum_{i=1}^\infty \frac{1}{T(\lambda_{i, d}^{-1},1,p,0) }\le  T(\varepsilon^{-1},1,p,0) \\
   & \implies   \frac{1}{T(\lambda_{n+1, d}^{-1},1,p,0)} \le \frac{1}{T(\varepsilon^{-1},1,p,0)}.
\end{align*}

This implies an upper bound on the complexity of
\begin{align*}
       \MoveEqLeft{\comp(\varepsilon,d)} \\
       &\le 1 + T(\varepsilon^{-1},1,p,0) \sum_{i=1}^\infty \frac{1}{T(\lambda_{i, d}^{-1},1,p,0)} \\
       & = 1 + T(\varepsilon^{-1},1, p,0) \left [ \sum_{i=1}^{L(p)-1} \frac{1}{T(\lambda_{i, d}^{-1},1,p,0)}
       + \underbrace{\sum_{i=L(p)}^\infty \frac{1}{T(\lambda_{i, d}^{-1},1,p,0)}}_{=:S_p <\infty\ \text{by \eqref{eq:strong_tractiff3}} }\right] \\
       & \le 1+ T(\varepsilon^{-1},1,p,0) \underbrace{\left[\frac{(L(p)-1)}{T(\lambda_{\max}^{-1},1,p,0)} + S_p\right]}_{=:C_2(p)}\\
       & = 1+ T(\varepsilon^{-1},1,p,0) C_2 (p)\\
       &\le C_{p,0} T(\varepsilon^{-1},1,p,0),
\end{align*}
for some suitably chosen $C_{p,0}>0$. This means that we have strong $T$-tractability, and verifies sufficiency of \eqref{eq:strong_tractiff3}. 
As for $\mathcal{P}_{\text{opt}}$, we immediately see from the above argument that $\mathcal{P}_{\text{opt}}=(p^*,0)$, where
\[
 p^* \le \inf\{p>0\ \mbox{such that \eqref{eq:strong_tractiff3} holds}\}.
\]


\bigskip
\noindent \textbf{Necessary condition:} \\
Suppose that we have strong 
$T$-tractability. That is, for some $p>0$ there exists a positive constant $C_{p,0}$ such that
\[
\comp(\varepsilon,d)\le C_{p,0}\, T(\varepsilon^{-1},1,p,0)
\]
for all $\varepsilon\in (0,\infty)$.


% This, in turn, means that 
% \begin{equation}\label{eq_strong_tract_p'3}
% \comp(\varepsilon,d)\le C_{p,0} T(\varepsilon^{-1},1,p',0)
% \end{equation}
% for all $p'>p$. 
% 
% \noindent This means for (possibly small) $\delta_p>0$, there exists a constant $K$ such that \eqref{eq_strong_tract_p'3} holds for $p'=p+\delta_p$.

Since the eigenvalues $\lambda_{j,d}$ are non-increasing, we have
\begin{equation}\label{eq:lambda_K_strong3}
\lambda_{\lfloor C_{p,0}\, T(\varepsilon^{-1},1,p,0)\rfloor +1,d}\le \varepsilon.
\end{equation}
Let
\[
j=j (\varepsilon):= \lfloor C_{p,0}\, T(\varepsilon^{-1},1,p,0)\rfloor +1.
\]
When we vary $\varepsilon\in (0,\infty)$, we have that $j\in \{j_d^*, j_d^*+1, j_d^*+2,\ldots\}$, where 
\[
  j_d^*=\lfloor  T(0,1,p,0)\rfloor +1,
\]
by the monotonicity of $T$.

Note furthermore that we always have
\[
j=j(\varepsilon)\le C_{p,0}\, T(\varepsilon^{-1},1,p,0)+1.
\]
Due to our assumptions, we know that 
$T(\cdot,1, p,0)$ is increasing in the first argument if 
we keep $p$ fixed. 

Then, by \eqref{eq:lambda_K_strong3},
\[
j \le C_{p,0} T(\varepsilon^{-1},1,p,0)+1\le C_{p,0} T(\lambda_{j,d}^{-1},1,p,0)+1,
\]
and the latter chain of inequalities holds for all $j\ge j_d^*$. 


This implies
\[ 
 T(\lambda_{j,d}^{-1},1, p,0) \ge \frac{j-1}{C_{p,0}}.
\]
This, in turn, implies, for any $\tau>1$, that 
\[ 
 (T(\lambda_{j,d}^{-1},1, p,0))^\tau \ge \frac{(j-1)^\tau}{C_{p,0}^\tau },
\]
which yields, due to our assumption on $T$, 
\[ 
 K_{p,0,\tau}\,T (\lambda_{j,d}^{-1},1,\tau p,0) \ge \frac{(j-1)^\tau}{C_{p,0}^\tau},
\]
for some $K_{p,0,\tau}>0$.
This gives
\[
\sum_{j=j_d^*}^\infty \frac{1}{T (\lambda_{j,d}^{-1},1, \tau p,0)} \le K_{p,0,\tau}\, C_{p,0}^\tau
\sum_{j=j_d^*}^\infty \frac{1}{(j-1)^\tau} \le
\peter{K_{p,0,\tau}\, C_{p,0}^\tau\, \zeta (\tau) }< \infty,
\]
\peter{where by $\zeta(\cdot)$ we denote the Riemann zeta function. Consequently, 
\[
\sup_{d\in\naturals} \sum_{j=j_d^*}^\infty \frac{1}{T (\lambda_{j,d}^{-1},1, \tau p,0)} \le K_{p,0,\tau}\, C_{p,0}^\tau\, \zeta (\tau) < \infty.
\]
}


Note, however, that, \peter{for any $d\in\naturals$,}
\[
 j_d^* = \lfloor C_{p,0}\, T(0,1,p,0)\rfloor +1 \le \lfloor C_{p,0}\, T(0,1,\tau p,0)\rfloor +1
\]
Hence, we can find $L(\tau p)$, which is independent of $\varepsilon$ and $d$, such that 
\[
 \peter{\sup_{d\in\naturals}} \sum_{j=L(\tau p)}^\infty \frac{1}{T(\lambda_{j,d}^{-1},1, \tau p,0)}<\infty.
\]
This yields \eqref{eq:strong_tractiff3} with $p$ replaced by $\tau p$, and so we see necessity of $\eqref{eq:strong_tractiff3}$. 


Assume that $\mathcal{P}_{\text{opt}}=(p^*,0)$. We already know from the first part of the proof that 
\[
 p^* \le \inf\{p>0\ \mbox{such that \eqref{eq:strong_tractiff3} holds}\}.
\]
Suppose that we would have 
\[
 p^* < \inf\{p>0\ \mbox{such that \eqref{eq:strong_tractiff3} holds}\}.
\]
\peter{Then, however, there exists a $p^\dag$ such that
\[
p^* < p^\dag < \inf\{p>0\ \mbox{such that \eqref{eq:strong_tractiff3} holds}\},
\]
and that the problem is $T$-tractable with parameters $q=0$ and $p^\dag$. }
We can proceed as in the second part of the proof and show that 
\[
 \sup_{d\in\naturals} \sum_{j=L(\tau \peter{p^\dag})}^\infty \frac{1}{T(\lambda_{j,d}^{-1},1, \tau \peter{p^\dag},0)}<\infty.
\]
for any $\tau>1$, and a suitable $L(\tau \peter{p^\dag})$. By choosing $\tau$ sufficiently close to 1, we then would 
obtain 
\[
 \tau \peter{p^\dag} < \inf\{p>0\ \mbox{such that \eqref{eq:strong_tractiff3} holds}\},
\]
which is a contradiction. So, we must have 
\[
 p^* = \inf\{p>0\ \mbox{such that \eqref{eq:strong_tractiff3} holds}\}.
\]

This concludes the proof of the theorem. 

\end{proof}
\bigskip

\begin{theorem}\label{thm_main_tract2} 
A problem is $T$-tractable iff there exist $p,q>0$, and $q_1>0$ and a constant $L(p,q_1) > 0$ such that
\begin{equation} \label{eq:tractiff4}
     S_{p,q,q_1}:=\sup_{d \in \naturals} 
     \sum_{i = \lceil L(p,q_1)\, T(0,d,p,q_1) \rceil}^\infty \frac{1}{T(\lambda^{-1}_{i,d},d,p,q)}< \infty.
\end{equation}
\end{theorem}

\peter{(PETER: Do we also want to include a statement on the optimal parameters $p^*,q^*$?)}

\begin{proof}
    \textbf{Sufficient Condition:}\\ Suppose that \eqref{eq:tractiff4} holds for some $p,q>0$. 
Because $T$ is increasing in its arguments, it follows that the complexity may be equivalently expressed as 
\begin{align*}
    \comp(\varepsilon, d) &= \min \{n \in \natzero : \lambda_{n+1, d} \le \varepsilon\} \\
    &= \min \biggl\{n \in \natzero : \frac{1}{T(\lambda_{n+1, d}^{-1},d,p,q)} \le \frac{1}{T(\varepsilon^{-1},d,p,q)} \biggr\}.
\end{align*}

Since the $\lambda_{i,d}$ are non-increasing it follows that the $\phi(\lambda_{i, d}^{-1},p)$ are non-decreasing.  
In particular, for all $n \in \naturals$,
\begin{align*}
    \MoveEqLeft{\lambda_{n+1,d} \le \lambda_{n,d} \le \cdots \le \lambda_{1,d}} \\
    & \implies \frac{1}{T(\lambda_{n+1, d}^{-1},d,p,q)} \le \frac{1}{T(\lambda_{n, d}^{-1},d,p,q))} \le \cdots \le \frac{1}{T(\lambda_{1, d}^{-1},d,p,q)} \\
    & \implies \frac{1}{T(\lambda_{n+1, d}^{-1},d,p.q)} 
    \le \frac 1n \sum_{i=1}^n  \frac{1}{T(\lambda_{i, d}^{-1},d,p,q)} 
    \le \frac 1n \sum_{i=1}^\infty  \frac{1}{T(\lambda_{i, d}^{-1},d,p,q)}.
\end{align*}
Thus, we can conclude that 
\begin{align*}
    n \ge T(\varepsilon^{-1},d,p,q) \sum_{i=1}^\infty \frac{1}{T(\lambda_{i, d}^{-1},d,p,q)}
    & \implies 
  \frac 1n \sum_{i=1}^\infty \frac{1}{T(\lambda_{i, d}^{-1},d,p,q)} \le  \frac{1}{T(\varepsilon^{-1},d,p,q)} \\
   & \implies   \frac{1}{T(\lambda_{n+1, d}^{-1},d,p,q)} \le \frac{1}{T(\varepsilon^{-1},d,p,q)}.
\end{align*}
This implies an upper bound on the complexity of
\begin{align*}
       \MoveEqLeft{\comp(\varepsilon,d)} \\
       &\le 1 + T(\varepsilon^{-1},d,p,q) \sum_{i=1}^\infty \frac{1}{T(\lambda_{i, d}^{-1},d,p,q)} \\       
       & = 1 + T(\varepsilon^{-1},d,p,q) \left [ \sum_{i=1}^{\lceil L(p,q_1)\, T(0,d,p,q_1)\rceil -1} \frac{1}{T(\lambda_{i, d}^{-1},d,p,q)}  
       + \sum_{i=\lceil L(p,q_1)\, T(0,d,p,q_1) \rceil}^\infty \frac{1}{T(\lambda_{i, d}^{-1},d,p,q)} \right ] \\
       & \le 1+ T(\varepsilon^{-1},d,p,q)\left[ \frac{\lceil L(p,q_1)\, T(0,d,p,q_1)\rceil-1}{T(\lambda_{\max}^{-1},d,p,q)} + S_{p,q,q_1}\right]
        \qquad \text{by \eqref{eq:tractiff4}} \\
       & \le 1+ T(\varepsilon^{-1},d,p,q)\left[L(p,q_1)\, T(0,d,p,q_1) + S_{p,q,q_1}\right]\\ 
       & \le 1+L(p,q_1)\,T(\varepsilon^{-1},d,p,q) T(0,d,p,q_1) + S_{p,q,q_1}\, T(\varepsilon^{-1},d,p,q)\\
       & \le 1+L(p,q_1)\,T(\varepsilon^{-1},d,p,q) T(\varepsilon^{-1},d,p,q_1) + S_{p,q,q_1}\, T(\varepsilon^{-1},d,p,q)\\
       & \le 1+L(p,q_1)\,T(\varepsilon^{-1},d,2p,2\max\{q,q_1\})  + S_{p,q}\, T(\varepsilon^{-1},d,p,q)  \\
       & \le C_{2p,2\max\{q,q_1\}} \,T(\varepsilon^{-1},d,2p,2\max\{q,q_1\})
\end{align*}
for some suitably chosen $C_{2p,2\max\{q,q_1\}}>0$. It follows that we have $T$-tractability, which shows sufficiency of \eqref{eq:tractiff4}. \\



\textbf{Necessary condition:}\\
Suppose that we have 
$T$-tractability. That is, for some $p>0$ and $q>0$, there exists a positive constant $C_{p,q}$ such that 
\[
\comp(\varepsilon,d)\le C_{p,q}\, (T(\varepsilon^{-1},d,p,q)
\]
for all $\varepsilon\in (0,\infty)$ and all $d\in\naturals$.


We know that
\[
\lambda_{\comp(\varepsilon,d)+1,d}\le \varepsilon.
\]
Since the eigenvalues $\lambda_{j,d}$ are non-increasing, we have
\begin{equation}\label{eq:lambda_K4}
\lambda_{\lfloor C_{p,q}\, T(\varepsilon^{-1},d,p,q)\rfloor +1,d}\le \varepsilon.
\end{equation}
Let
\[
j=j (\varepsilon):= \lfloor C_{p,q}\, T(\varepsilon^{-1},d,p,q)\rfloor +1.
\]
If we vary $\varepsilon\in (0,\infty)$, we see that $j=j_d^*, j_d^*+1, j_d^*+2,\ldots$, where 
\[
  j_d^*=\lfloor C_{p,q}\, T(0,d,p,q)\rfloor +1
\]
Note that
\[
j\le C_{p,q}\, T(\varepsilon^{-1},d,p,q) +1.
\]
Due to our assumptions, we know that 
$T(\cdot,d,p,q)$ is increasing in the first argument if 
we keep $p$ and $q$ fixed.  

Then, by \eqref{eq:lambda_K4},
\[
j \le C_{p,q}\, T(\varepsilon^{-1},d,p,q) +1 \le C_{p,q}\, T(\lambda_{j,d}^{-1},d,p,q) +1,
\]
and the latter chain of inequalities holds for all $j\ge j_d^*$.

This implies
\[ 
T(\lambda_{j,d}^{-1},d, p,q) \ge \frac{j-1}{C_{p,q}}.
\]
This, in turn, implies, for any $\tau>1$, that 
\[ 
 (T(\lambda_{j,d}^{-1},d,p,q))^\tau \ge \frac{(j-1)^\tau}{C_{p,q}^\tau},
\]
which yields
\[ 
 K_{p,q,\tau}\, T (\lambda_{j,d}^{-1},d,\tau p,\tau q) \ge (T(\lambda_{j,d}^{-1},d, p,q))^\tau 
 \ge \frac{(j-1)^\tau}{C_{p,q}^\tau}.
\]
This then implies
\[
\sum_{j=j_d^*}^\infty \frac{1}{T(\lambda_{j,d}^{-1},d, \tau p,\tau q)} 
\le K_{p,q,\tau}\, C_{p,q}^\tau \sum_{j=j_d^*}^\infty \frac{1}{(j-1)^\tau}
\peter{\le K_{p,q,\tau}\, C_{p,q}^\tau \, \zeta(\tau)}
<\infty.
\]
\peter{
Consequently,
\[
\sup_{d\in\naturals} \sum_{j=j_d^*}^\infty \frac{1}{T(\lambda_{j,d}^{-1},d, \tau p,\tau q)} \le K_{p,q,\tau}\, C_{p,q}^\tau \, \zeta(\tau) < \infty.
\]
}
Note now that, \peter{for any $d\in\naturals$,}
\[
 j_d^* = \lfloor C_{p,q}\, T(0,d,p,q)\rfloor +1 \le \lfloor C_{p,q}^\tau\, T(0,d,\tau p,\tau q)\rfloor +1.
\]
So, there exists $L(\tau p,\tau q)$ such that 
\[
\peter{\sup_{d\in\naturals}}\,\, \sum_{j=\lceil L(\tau p,\tau q) T(0,d,\tau p,\tau q)\rceil}^\infty \frac{1}{T(\lambda_{j,d}^{-1},d, \tau p,\tau q)}<\infty.
\]


This yields \eqref{eq:tractiff4} with $p$ replaced by $\tau p$ and $q$ and $q_1$ replaced by $\tau q$, and so we see the necessity of $\eqref{eq:tractiff4}$. 

\end{proof}


\section{Examples}

\paragraph{Example 1: Strong polynomial tractability}

Let $T: (0,\infty) \times \naturals \times [0,\infty) \times [0,\infty)\Rightarrow [1,\infty)$ 
be defined by 
\[
 T(\varepsilon^{-1},d,p,q)= (\max\{\varepsilon^{-1},1\})^p .
\]
Then Theorem \ref{thm_main_strong_tract2} yields the iff-condition
\[
 \sup_{d\in\naturals} \sum_{i=L(p)}^\infty (\min\{\lambda_{i,d},1\})^p < \infty.
\]
This essentially recovers the result on strong polynomial tractability in \cite[Theorem 5.1]{NW08}

\paragraph{Example 2: Polynomial tractability}

Let $T: (0,\infty) \times \naturals \times [0,\infty) \times [0,\infty)\Rightarrow [1,\infty)$ 
be defined by 
\[
 T(\varepsilon^{-1},d,p,q)= (\max\{\varepsilon^{-1},1\})^p\, d^q.
\]
Then Theorem \ref{thm_main_tract2} yields the iff-condition
\[
 \sup_{d\in\naturals} \sum_{i=L(p,q_1) d^{q_1}}^\infty (\min\{\lambda_{i,d},1\})^p d^{-q} < \infty.
\]
This essentially recovers the result on polynomial tractability in \cite[Theorem 5.1]{NW08}



\peter{(PETER: Add further examples (e.g., EXP- polynomial tractability, etc.)}

\bigskip

\fred{We need to check the proof for 
\begin{itemize}
\item correctness
\item exponents
\item whether the additional assumptions are too restrictive and can be generalized
\item whether we need $\tau$ in the assumption
\item Do we need non-decreasing or strictly increasing? 
\end{itemize}}
\begin{definition}[Weak Tractability]
If $n(\varepsilon,d)$ denotes the minimal number of function values needed to approximate all functions. We say that the problem is \emph{weakly tractable} if
    \[\lim_{\varepsilon^{-1}+d\rightarrow \infty} \frac{\ln n(\varepsilon,d)}{\varepsilon^{-1}+d} = 0\]
\end{definition}
\begin{definition}[Quasi-polynomial Tractability]
    The problem is \emph{quasi-polynomially tractable}, if there exist two constants $C, t> 0$ such that 
    \[
    n(\varepsilon,d) \leq C\exp\{t(1+\ln(1/\varepsilon))(1+\ln(d))\}
    \] for all $\varepsilon \in (0,1)$ and $d\in\mathbb{N}$.
\end{definition}
\begin{thebibliography}{99}

\bibitem{NW08} E.~Novak,H.~Wo\'zniakowski. \textit{Tractability of Multivariate Problems, Volume I: Linear Information}. 
European Mathematical Society Publishing House, Zurich, 2008.
 ×
\end{thebibliography}


\end{document}
