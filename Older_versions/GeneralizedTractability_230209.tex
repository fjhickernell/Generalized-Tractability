\documentclass[11pt,a4paper]{article}
\usepackage{latexsym,amsfonts,amsmath,graphics,mathtools,amssymb,amsthm,booktabs}
\usepackage{epsfig}
%\usepackage[notref,notcite]{showkeys}
%\usepackage[active]{srcltx}
\usepackage{enumitem}
\usepackage[usenames,dvipsnames,svgnames,table]{xcolor}

\usepackage{algorithmic}

%\input{common-macros}
%\input{common-macros2}

\input{FJHDef.tex}



\setlength{\textheight}{24cm} \setlength{\textwidth}{16cm}
\setlength{\hoffset}{-1.3cm} \setlength{\voffset}{-1.8cm}

\newcommand{\supp}{\operatorname{supp}}

\newcommand{\tpmod}[1]{{\;(\operatorname{mod}\;#1)}}

\DeclareMathOperator*{\esssup}{ess\,sup}

%\DeclareMathOperator{\comp}{comp}
\DeclareMathOperator{\SOL}{SOL}
\DeclareMathOperator{\APP}{APP}

\allowdisplaybreaks

\newcommand{\fred}[1]{\begingroup\color{blue}Fred: #1\endgroup}
\newcommand{\peter}[1]{\begingroup\color{purple}Peter: #1\endgroup}
\newcommand{\kachi}[1]{\begingroup\color{ForestGreen}#1\endgroup}


\begin{document}

\newtheorem{theorem}{Theorem}
\theoremstyle{definition}
\newtheorem{definition}{Definition}


\title{Generalized Tractability}
\author{\fred{Fred J. Hickernell}, \peter{Peter Kritzer}, \kachi{Onyekachi Osisiogu}, and \\ Henryk Wo\'zniakwoski}
\date{\today}

\maketitle

\section{Introduction}

TODO: We need to check the relation of our work to that of Michael Gnewuch and Henryk. We do not need tensor product structure: do they need that assumption? The summation conditions on the eigenvalues might be a significant difference between our results and theirs.

\medskip

TODO: Check in how far our conditions coincide with existing conditions for less general tractability notions (polynomial tractability, etc.).

\medskip

Let $\cf_d$ and $\cg_d$ Hilbert spaces, and $\SOL_d : \cf_d \to \cg_d$ be a linear solution operator with adjoint $\SOL_d^*$ such that $\SOL_d^*\SOL_d : \cf_d \to \cf_d$ has eigenvalues and orthonormal eigenvectors  
\[
\lambda_{\max}^2 \ge \lambda_{1,d}^2 \ge \lambda_{2,d}^2 \ge \cdots, \qquad u_{1,d}, u_{2,d}, \ldots, 
\]
This means that the non-negative $\lambda_{i,d}$ are the singular values of the $\SOL_d$. The goal is to find an approximate solution, $\APP_d:\cb_d \times (0,\infty) \to \cg_d$ satisfying 
\[
\norm[\cg_d]{\SOL_d(f) - \APP_d(f,\varepsilon)} \le \varepsilon,
\]
where $\cb_d$ is the unit ball in $\cf_d$ and $\APP_d(f,\varepsilon)$ is allowed to depend on arbitrary linear functionals.  In this case, the optimal solution is 
\[
\APP_d(f,\varepsilon) = \sum_{i=1}^n \SOL(u_i) \ip[\cf_d]{f}{u_i},
\]
and the complexity of our linear problem is
\[
\comp(\varepsilon, d) = \min \{n \in \natzero : \lambda_{n+1, d} \le \varepsilon\}.
\]

Certain kinds of common notions of tractability are 
\begin{equation*}
    \begin{array}{r@{\qquad}c@{\qquad}c}
    & \comp(\varepsilon, d)\\
    \toprule
    \text{strong polynomial algebraic  tractability} & \Theta(\varepsilon^{-p})\\
    \text{polynomial algebraic tractability} & \Theta(\varepsilon^{-p}d^{q})\\
    \text{strong polynomial exponential tractability} &  \Theta\bigl([\log(1 + \varepsilon^{-1})]^p\bigr)\\
    \text{polynomial exponential tractability} & \Theta\bigl([\log(1 + \varepsilon^{-1})]^p  d^{q}\bigr)
    \end{array}
\end{equation*}
Let $\phi : (0,\infty) \times (0,\infty) \to (0,\infty)$ and $\psi : \naturals  \times (0,\infty) \to (0,\infty)$ be positive-valued functions that are strictly increasing in both arguments, e.g., 
\[
(x,p) \mapsto x^{p}, \qquad (x,p) \mapsto [\log(1+x)]^p.
\]

All the examples of tractability above are of the form $\comp(\varepsilon, d) = \Theta \bigl( \phi(\varepsilon^{-1},p) \psi(d,q)\bigr)$.  
Note, we define the $\Theta$ notation as follows
\begin{equation*}
     g(\varepsilon,d) = \Theta\bigl(\phi(\varepsilon^{-1},p) \bigr) \text{ means } 
    \lim_{\varepsilon \to 0} \frac{g(\varepsilon,d)}{\phi(\varepsilon^{-1},p')}
           \begin{cases} < \infty, & p'> p,  \\ 
          =\infty, & p' < p.\end{cases}
\end{equation*}
\begin{multline*}
     g(\varepsilon,d) = \Theta\bigl(\phi(\varepsilon^{-1},p) \psi(d,q) \bigr) \text{ means } \\
    \lim_{\substack{\varepsilon \to 0 \\ d \to \infty}} \frac{g(\varepsilon,d)}{\phi(\varepsilon^{-1},p') \psi(d,q')}
          \begin{cases} < \infty, & p'> p, \  q' > q,  \\ 
          =\infty, & p' < p, \ q' < q.\end{cases}
\end{multline*}


\begin{definition}
If there exist the functions $\phi$ and $\psi$ satisfying the conditions listed above and there exist positive numbers $p$ and $q$ with
\[
\comp(\varepsilon, d)  = \Theta\bigl(\phi(\varepsilon^{-1},p) \psi(d,q) \bigr),
\]
then we call a problem $\phi$-$\psi$-tractable with exponents $p$ and $q$.  
If there exists a function $\phi$ satisfying the conditions above and a positive number $p$ such that 
\begin{equation}\label{eq:def_strong_tract}
\comp(\varepsilon, d)  = \Theta\bigl(\phi(\varepsilon^{-1},p) \bigr)
\end{equation}
then we call a problem strongly $\phi$-tractable with exponent $p$. The infimum $p^*$ of all exponents $p>0$ such that \eqref{eq:def_strong_tract} holds, 
is called the exponent of strong $\phi$-tractability. 
\end{definition}


\section{Main Results}
Suppose that, for any real $\tau_1, \tau_2>0$ and any $p,q>0$, it holds that
\begin{align*}
 (\phi(\varepsilon^{-1},p))^{\tau_1} &\le K_{p,\tau_1} \phi (\varepsilon^{-1},\tau_1 p)\\
  (\psi(d,q))^{\tau_2} &\le L_{q,\tau_2} \psi (d,\tau_2 q)\\
\end{align*}
for all $\varepsilon\in (0,1)$, where $K_{p,\tau_1}>0$ is a constant that may depend on $p$ and $\tau_1$, but is independent of $\varepsilon$, 
and where $L_{q,\tau_2}>0$ is a constant that may depend on $q$ and $\tau_2$, but is independent of $d$.

We first show the following result. 
\begin{theorem}\label{thm_main_strong_tract} 
A problem is strongly $\phi$-tractable iff there exists $p>0$ and an integer $C(p)\ge 0$ such that
\begin{equation} \label{eq:strong_tractiff}
     \sup_{d \in \naturals} \sum_{i = C(p)}^\infty \frac{1}{\phi(\lambda_{i, d}^{-1},p)} < \infty.
\end{equation}
If \eqref{eq:strong_tractiff} holds, then the exponent of strong $\phi$-tractability equals
\[
 p^* := \inf\{p>0\ \mbox{such that \eqref{eq:tractiff} holds}\}.
\]

\end{theorem}

\begin{proof}
We first show the sufficiency of \eqref{eq:strong_tractiff}. Suppose that \eqref{eq:strong_tractiff} holds for some $p>0$.  
Because $\phi$ is increasing in its arguments, it follows that the complexity may be equivalently expressed as 
\begin{align*}
    \comp(\varepsilon, d) &= \min \{n \in \natzero : \lambda_{n+1, d} \le \varepsilon\} \\
    &= \min \biggl\{n \in \natzero : \frac{1}{\phi(\lambda_{n+1, d}^{-1},p)} \le \frac{1}{\phi(\varepsilon^{-1},p)} \biggr\}.
\end{align*}
Since the $\lambda_{i,d}$ are non-increasing it follows that the $\phi(\lambda_{i, d}^{-1},p)$ are non-decreasing. 
In particular, for all $n \in \naturals$,
\begin{align*}
    \MoveEqLeft{\lambda_{n+1,d} \le \lambda_{n,d} \le \cdots \le \lambda_{1,d}} \\
    & \implies \frac{1}{\phi(\lambda_{n+1, d}^{-1},p)} \le \frac{1}{\phi(\lambda_{n, d}^{-1},p)} \le \cdots \le \frac{1}{\phi(\lambda_{1, d}^{-1},p)} \\
    & \implies \frac{1}{\phi(\lambda_{n+1, d}^{-1},p)} 
    \le \frac 1n \sum_{i=1}^n  \frac{1}{\phi(\lambda_{i, d}^{-1},p)} 
    \le \frac 1n \sum_{i=1}^\infty  \frac{1}{\phi(\lambda_{i, d}^{-1},p)}.
\end{align*}
Thus, we can conclude that 
\begin{align*}
    n \ge \phi(\varepsilon^{-1},p) \sum_{i=1}^\infty \frac{1}{\phi(\lambda_{i, d}^{-1},p)}
    & \implies 
  \frac 1n \sum_{i=1}^\infty \frac{1}{\phi(\lambda_{i, d}^{-1},p)} \le  \frac{1}{\phi(\varepsilon^{-1},p')} \\
   & \implies   \frac{1}{\phi(\lambda_{n+1, d}^{-1},p)} \le \frac{1}{\phi(\varepsilon^{-1},p)}.
\end{align*}
This implies an upper bound on the complexity of
\begin{align*}
       \MoveEqLeft{\comp(\varepsilon,d)} \\
       &\le 1 + \phi(\varepsilon^{-1},p) \sum_{i=1}^\infty \frac{1}{\phi(\lambda_{i, d}^{-1},p)} \\
       & = 1 + \phi(\varepsilon^{-1}, p) \left [ \sum_{i=1}^{C(p)-1} \frac{1}{\phi(\lambda_{i, d}^{-1},p)}  
       + \underbrace{\sum_{i=C(p)}^\infty \frac{1}{\phi(\lambda_{i, d}^{-1},p)}}_{=:S_p <\infty\ \text{by \eqref{eq:tractiff}} }\right] \\
       & \le 1+ \phi(\varepsilon^{-1},p) \underbrace{\left(\frac{C(p)-1}{\phi(\lambda_{\max}^{-1},p)} + S_p\right)}_{=:C_1(p)}\\
       & = 1+ \phi(\varepsilon^{-1},p) C_1 (p),
\end{align*}
which means that we have strong $\phi$-tractability, and verifies sufficiency of \eqref{eq:strong_tractiff}. 

\bigskip

For the exponent $p^*$ of $\phi$-tractability, we then obviously obtain 
\[
 p^* \le \inf\{p>0\ \mbox{such that \eqref{eq:strong_tractiff} holds}\}.
\]


\bigskip


Let us now show the necessity of \eqref{eq:strong_tractiff}. Indeed, suppose that we have strong 
$\phi$-tractability. I.e., 
\[
\comp(\varepsilon,d)=\Theta (\phi(\varepsilon^{-1},p))
\]
for some $p>0$. This, in turn, means that there is a positive constant $K$ such that
\begin{equation}\label{eq_strong_tract_p'}
\comp(\varepsilon,d)\le K \phi(\varepsilon^{-1},p')
\end{equation}
for all $p'>p$. 

Hence, for (possibly small) $\delta_p>0$, there exists a constant $K$ such that \eqref{eq_strong_tract_p'} holds for $p'=p+\delta_p$.

We know that
\[
\lambda_{\comp(\varepsilon,d)+1,d}\le \varepsilon.
\]
Since the eigenvalues $\lambda_{j,d}$ are non-increasing, we have
\begin{equation}\label{eq:lambda_K_strong3}
\lambda_{\lfloor K \phi(\varepsilon^{-1},p')\rfloor +1,d}\le \varepsilon.
\end{equation}
Let
\[
j=j (\varepsilon):= \lfloor K \phi(\varepsilon^{-1},p')\rfloor +1.
\]
If we vary $\varepsilon\in (0,1]$, we see that $j=j_d^*, j_d^*+1, j_d^*+2,\ldots$, where 
\[
  j_d^*=\lfloor K \phi(1,p')\rfloor +1
\]
Note that
\[
j\le K \phi(\varepsilon^{-1},p')+1.
\]
Due to our assumptions, we know that 
$\phi(\cdot, p')$ is increasing in the first argument if 
we keep $p'$ fixed. Let $\phi_{p'}^{-1} (\cdot)$ be the inverse function of $\phi (\cdot, p')$ with respect to the first argument. 
Note that also $\phi_{p'}^{-1}$ is strictly increasing. 

Then, 
\[
j \le K \phi(\varepsilon^{-1},p')+1
\]
is equivalent to
\[
\frac{j-1}{K} \le \phi(\varepsilon^{-1},p').
\]
By the monotonicity of $\phi_{p'}^{-1}$, the latter inequality holds if and only if
\[
  \phi_{p'}^{-1} \left(\frac{j-1}{K} \right)
  \le \varepsilon^{-1},
\]
or, equivalently,
\[
 \varepsilon \le \frac{1}{\phi_{p'}^{-1} \left(\frac{j-1}{K} \right)}.
\]
By using \eqref{eq:lambda_K_strong3}, we derive
\[
 \lambda_{j,d}\le \frac{1}{\phi_{p'}^{-1} \left(\frac{j-1}{K} \right)}
\]
for all $j\ge j_d^*$, which is again equivalent to 
\[
\lambda_{j,d}^{-1}\ge \phi_{p'}^{-1} \left(\frac{j-1}{K} \right),
\]
which holds if and only if
\[ 
 \phi (\lambda_{j,d}^{-1}, p') \ge \frac{j-1}{K}.
\]
This implies, for any $\tau>1$, that 
\[ 
 (\phi (\lambda_{j,d}^{-1}, p'))^\tau \ge \frac{(j-1)^\tau}{K^\tau },
\]
which implies, due to our assumption on $\phi$, 
\[ 
 K_{p',\tau}\,\phi (\lambda_{j,d}^{-1}, \tau p') \ge \frac{(j-1)^\tau}{K^\tau }.
\]
This yields
\[
\sum_{j=j_d^*}^\infty \frac{1}{\phi (\lambda_{j,d}^{-1}, \tau p')} \le K_{p',\tau}\, K^\tau \sum_{j=j_d^*}^\infty \frac{1}{(j-1)^\tau}<\infty.
\]

Note, however, that 
\[
 j_d^* = \lfloor K \phi(1,p')\rfloor +1 \le \lfloor K \phi(1,\tau p')\rfloor +1
\]
Hence, we can find $C(\tau p')$ such that 
\[
 \sum_{j=C(\tau p')}^\infty \frac{1}{\phi (\lambda_{j,d}^{-1}, \tau p')}<\infty.
\]
This yields \eqref{eq:strong_tractiff} with $p$ replaced by $\tau p'$, and so we see necessity of $\eqref{eq:strong_tractiff}$. 

\bigskip

Assume that $p^*$ is the $\varepsilon$-exponent of strong $\phi$-tractability. We already know from the first part of the proof that 
\[
 p^* \le \inf\{p>0\ \mbox{such that \eqref{eq:strong_tractiff} holds}\}.
\]
Suppose that we would have 
\[
 p^* < \inf\{p>0\ \mbox{such that \eqref{eq:strong_tractiff} holds}\}.
\]
Then, however, we could proceed as in the second part of the proof and show that 
\[
 \sum_{j=C(\tau p')}^\infty \frac{1}{\phi (\lambda_{j,d}^{-1}, \tau p')}<\infty.
\]
for any $\tau>1$, any $p'>p^*$, and a suitable $C(\tau p')$. By choosing $\tau$ sufficiently close to 1, and $p'$ sufficiently close to $p^*$, we then would 
obtain 
\[
 \tau p' < \inf\{p>0\ \mbox{such that \eqref{eq:strong_tractiff} holds}\},
\]
which is a contradiction. So, we must have 
\[
 p^* = \inf\{p>0\ \mbox{such that \eqref{eq:strong_tractiff} holds}\}.
\]

This concludes the proof of the theorem. 
\end{proof}

Next, we show a theorem characterizing $\phi$-$\psi$-tractability. 
\begin{theorem}\label{thm_main_tract} 
A problem is $\phi$-$\psi$-tractable iff there exist $p,q>0$ and an integer $C(p,q) \ge 0$ such that
\begin{equation} \label{eq:tractiff}
     \sup_{d \in \naturals} \frac{1}{\psi(d,q)} \sum_{i = C(p,q)}^\infty \frac{1}{\phi(\lambda_{i, d}^{-1},p))} < \infty.
\end{equation}
\end{theorem}
\begin{proof}
We first show the sufficiency of \eqref{eq:tractiff}. Suppose that \eqref{eq:tractiff} holds for some $p,q>0$. 
Because $\phi$ is increasing in its arguments, it follows that the complexity may be equivalently expressed as 
\begin{align*}
    \comp(\varepsilon, d) &= \min \{n \in \natzero : \lambda_{n+1, d} \le \varepsilon\} \\
    &= \min \biggl\{n \in \natzero : \frac{1}{\phi(\lambda_{n+1, d}^{-1},p)} \le \frac{1}{\phi(\varepsilon^{-1},p)} \biggr\}.
\end{align*}

Since the $\lambda_{i,d}$ are non-increasing it follows that the $\phi(\lambda_{i, d}^{-1},p)$ are non-decreasing, and this holds for any $\tau>1$.  
In particular, for all $n \in \naturals$,
\begin{align*}
    \MoveEqLeft{\lambda_{n+1,d} \le \lambda_{n,d} \le \cdots \le \lambda_{1,d}} \\
    & \implies \frac{1}{\phi(\lambda_{n+1, d}^{-1},p)} \le \frac{1}{\phi(\lambda_{n, d}^{-1},p))} \le \cdots \le \frac{1}{\phi(\lambda_{1, d}^{-1},p)} \\
    & \implies \frac{1}{\phi(\lambda_{n+1, d}^{-1},p)} 
    \le \frac 1n \sum_{i=1}^n  \frac{1}{\phi(\lambda_{i, d}^{-1},p)} 
    \le \frac 1n \sum_{i=1}^\infty  \frac{1}{\phi(\lambda_{i, d}^{-1},p)}.
\end{align*}
Thus, we can conclude that 
\begin{align*}
    n \ge \phi(\varepsilon^{-1},p) \sum_{i=1}^\infty \frac{1}{\phi(\lambda_{i, d}^{-1},p)}
    & \implies 
  \frac 1n \sum_{i=1}^\infty \frac{1}{\phi(\lambda_{i, d}^{-1},p)} \le  \frac{1}{\phi(\varepsilon^{-1},p)} \\
   & \implies   \frac{1}{\phi(\lambda_{n+1, d}^{-1},p)} \le \frac{1}{\phi(\varepsilon^{-1},p)}.
\end{align*}
This implies an upper bound on the complexity of
\begin{align*}
       \MoveEqLeft{\comp(\varepsilon,d)} \\
       &\le 1 + \phi(\varepsilon^{-1},p) \sum_{i=1}^\infty \frac{1}{\phi(\lambda_{i, d}^{-1},p)} \\       
       & = 1 + \phi(\varepsilon^{-1},p) \left [ \sum_{i=1}^{C(p,q)-1} \frac{1}{\phi(\lambda_{i, d}^{-1},p)}  
       + \sum_{i=C(p,q)}^\infty \frac{1}{\phi(\lambda_{i, d}^{-1},p)} \right ] \\
       & \le 1+ \phi(\varepsilon^{-1},p) \underbrace{\frac{C(p,q)-1}{\phi(\lambda_{\max}^{-1},p)}}_{=:C_1(p,q)}
       +  \phi(\varepsilon^{-1},p) \psi(d,q)
        \qquad \text{by \eqref{eq:tractiff}} \\
        & \le 1+\phi(\varepsilon^{-1},p) \bigl[C_1(p,q) + \psi(d,q) \bigr].
\end{align*}
It follows that we have $\phi$-$\psi$-tractability, which shows sufficiency of \eqref{eq:tractiff}. 


\bigskip

%---------------------------------------------------------------------
%---------------------------------------------------------------------

Let us now show necessity. Indeed, suppose that we have 
$\phi$-$\psi$-tractability with parameters $p$ and $q$. I.e., 
\[
\comp(\varepsilon,d)=\Theta (\phi(\varepsilon^{-1},p)\psi(d,q))
\]
for some $p,q>0$. This, in turn, means that there is a positive constant $K$ such that
\begin{equation}\label{eq_tract_p'q'}
\comp(\varepsilon,d)\le K \phi(\varepsilon^{-1},p')\psi(d,q')
\end{equation}
for all $p'>p$ and $q'>q$. 

Hence, for (possibly small) $\delta_p>0$ and $\delta_q>0$, there exists a constant $K$ such that \eqref{eq_tract_p'q'} holds for $p'=p+\delta_p$ 
and $q'=q+\delta_q$.

We know that
\[
\lambda_{\comp(\varepsilon,d)+1,d}\le \varepsilon.
\]
Since the eigenvalues $\lambda_{j,d}$ are non-increasing, we have
\begin{equation}\label{eq:lambda_K}
\lambda_{\lfloor K \phi(\varepsilon^{-1},p')\psi(d,q')\rfloor +1,d}\le \varepsilon.
\end{equation}
Let
\[
j=j (\varepsilon):= \lfloor K \phi(\varepsilon^{-1},p')\psi(d,q')\rfloor +1.
\]
If we vary $\varepsilon\in (0,1]$, we see that $j=j_d^*, j_d^*+1, j_d^*+2,\ldots$, where 
\[
  j_d^*=\lfloor K \phi(1,p')\psi(d,q')\rfloor +1
\]
Note that
\[
j\le K \phi(\varepsilon^{-1},p')\psi(d,q') +1.
\]
Due to our assumptions, we know that 
$\phi(\cdot, p')$ is increasing in the first argument if 
we keep $p'$ fixed. Let $\phi_{p'}^{-1} (\cdot)$ be the inverse function of $\phi (\cdot, p')$ with respect to the first argument. Note that also $\phi_{p'}^{-1}$ is strictly increasing. 

Then, 
\[
j \le K \phi(\varepsilon^{-1},p')\psi(d,q') +1
\]
is equivalent to
\[
\frac{j-1}{K \psi(d,q')} \le \phi(\varepsilon^{-1},p').
\]
By the monotonicity of $\phi_{p'}^{-1}$, the latter inequality holds if and only if
\[
  \phi_{p'}^{-1} \left(\frac{j-1}{K \psi(d,q')} \right)
  \le \varepsilon^{-1},
\]
or, equivalently,
\[
 \varepsilon \le \frac{1}{\phi_{p'}^{-1} \left(\frac{j-1}{K \psi(d,q')} \right)}.
\]
By using \eqref{eq:lambda_K}, we derive
\[
 \lambda_{j,d}\le \frac{1}{\phi_{p'}^{-1} \left(\frac{j-1}{K \psi(d,q')} \right)}
\]
for all $j\ge j_d^*$, which is again equivalent to 
\[
\lambda_{j,d}^{-1}\ge \phi_{p'}^{-1} \left(\frac{j-1}{K \psi(d,q')} \right),
\]
which holds if and only if
\[ 
 \phi (\lambda_{j,d}^{-1}, p') \ge \frac{j-1}{K \psi(d,q')}.
\]
This implies, for any $\tau>1$, that 
\[ 
 (\phi (\lambda_{j,d}^{-1}, p'))^\tau \ge \frac{(j-1)^\tau}{K^\tau (\psi(d,q'))^\tau},
\]
which yields
\[ 
 K_{p',\tau}\, \phi (\lambda_{j,d}^{-1}, \tau p') \ge (\phi (\lambda_{j,d}^{-1}, p'))^\tau 
 \ge \frac{(j-1)^\tau}{K^\tau (\psi(d,q'))^\tau}\ge \frac{(j-1)^\tau}{K^\tau\, L_{q',\tau}\, \psi(d,\tau q')}.
\]
This then implies
\[
\frac{1}{L_{q',\tau}\, \psi(d,\tau q')} \sum_{j=j_d^*}^\infty \frac{1}{\phi (\lambda_{j,d}^{-1}, \tau p')} 
\le K_{p',\tau}\, K^\tau \sum_{j=j_d^*}^\infty \frac{1}{(j-1)^\tau}<\infty.
\]
Note now that
\[
 j_d^* = \lfloor K \phi(1,p')\psi (1,q')\rfloor +1 \le \lfloor K\phi(1,\tau p')\psi (1,\tau q')\rfloor +1.
\]
So, there exists $C(\tau p',\tau q')$ such that 
\[
 \sum_{j=C(\tau p',\tau q')}^\infty \frac{1}{\phi (\lambda_{j,d}^{-1}, \tau p')}<\infty.
\]
This yields \eqref{eq:tractiff} with $p$ replaced by $\tau p'$ and $q$ replaced by $\tau q'$, and so we see the necessity of $\eqref{eq:tractiff}$. 

\end{proof}
\newpage
\bigskip

\kachi{
Let $T$ be a tractability function defined as follows
\[
T :(0,\infty) \times \mathbb{N} \times [0,\infty) \times [0,\infty) \rightarrow [1,\infty)
\] 
strictly increasing in all variables.

Also,
\[
[T(\varepsilon^{-1},d,p,q)]^\tau \le K_{p,q,\tau} T(\varepsilon^{-1},d,\tau p,\tau q)
\]
A problem is $T$-tractable with parameters $p$ and $q$ iff
\begin{equation} \label{comp_def}
\comp(\varepsilon,d) \le C(p,q) T(\varepsilon^{-1},d,p,q) \qquad \forall \varepsilon \in (0,\infty), \ d \in \mathbb{N}
\end{equation}
The  set of optimal parameters, $\mathcal{P}_{\text{opt}}$ is defined as
\begin{multline}
    \mathcal{P}_{\text{opt}} : = \{(p^*,q^*) : \eqref{comp_def} \text{ holds }\forall (p,q) \in [p^*,\infty) \times [q^*,\infty) \setminus (p^*,q^*) \\ 
    \eqref{comp_def} \text{ fails } \forall (p,q) \in [0,p^*] \times [0,q^*] \setminus (p^*,q^*) \}
\end{multline}






\begin{theorem}\label{thm_main_tract2} 
A problem is $T$-tractable with $q=0$ iff there exists $p>0$ and an integer $C(p) \ge 0$ such that
\begin{equation} \label{eq:strong_tractiff3}
     \sup_{d \in \naturals} \sum_{i = C(p)}^\infty \frac{1}{T(\lambda_{i,d}^{-1},1,p,0)} < \infty.
\end{equation}
%$ \frac{1}{\psi(d,q)}\frac{1}{\phi(\lambda_{i, d}^{-1},p))} $
\end{theorem}

If \eqref{eq:strong_tractiff3} holds, then the optimal $p$
is defined as
\[
 p^* := \inf\{p>0\ \mbox{such that \eqref{comp_def} holds}\}.
\]
\bigskip
\begin{proof}
\textbf{Sufficient condition:}\\
Given that $T$ is increasing in its arguments, we have the following equivalent expression 
    \begin{align*}
    \comp(\varepsilon, d) &= \min \{n \in \natzero : \lambda_{n+1, d} \le \varepsilon\} \\
    &= \min \biggl\{n \in \natzero : \frac{1}{T(\lambda_{n+1,d}^{-1},1,p,0)}\le \frac{1}{T(\varepsilon^{-1},1,p,0) }\biggr\}.
\end{align*}
Also since the $\lambda_{i,d}$ are non-increasing, it follows that the $T(\lambda_{i,d}^{-1},1,p,0)$ are non-decreasing. In particular, for all $n\in \mathbb{N}$, we have

\begin{align*}
    \MoveEqLeft{\lambda_{n+1,d} \le \lambda_{n,d} \le \cdots \le \lambda_{1,d}} \\
    & \implies \frac{1}{T(\lambda_{n+1, d}^{-1},1,p,0)} \le \frac{1}{T(\lambda_{n, d}^{-1},1,p,0)} \le \cdots \le \frac{1}{T(\lambda_{1, d}^{-1},1,p,0)} \\
    & \implies \frac{1}{T(\lambda_{n+1, d}^{-1},1,p,0) }
    \le \frac 1n \sum_{i=1}^n  \frac{1}{T(\lambda_{i, d}^{-1},1,p,0) }
    \le \frac 1n \sum_{i=1}^\infty  \frac{1}{T(\lambda_{i, d}^{-1},1,p,0)}.
\end{align*}
Thus, we can conclude that 
\begin{align*}
    n \ge T(\varepsilon^{-1},1,p,0) \sum_{i=1}^\infty \frac{1}{T(\lambda_{i, d}^{-1},1,p,0)}
    & \implies 
\frac 1n \sum_{i=1}^\infty \frac{1}{T(\lambda_{i, d}^{-1},1,p,0) }\le  T(\varepsilon^{-1},1,p,0) \\
   & \implies   \frac{1}{T(\lambda_{n+1, d}^{-1},1,p,0)} \le \frac{1}{T(\varepsilon^{-1},1,p,0)}.
\end{align*}

This implies an upper bound on the complexity of
\begin{align*}
       \MoveEqLeft{\comp(\varepsilon,d)} \\
       &\le 1 + T(\varepsilon^{-1},1,p,0) \sum_{i=1}^\infty \frac{1}{T(\lambda_{i, d}^{-1},1,p,0)} \\
       & = 1 + T(\varepsilon^{-1},1, p,0) \left [ \sum_{i=1}^{C(p)-1} \frac{1}{T(\lambda_{i, d}^{-1},1,p,0)}
       + \underbrace{\sum_{i=C(p)}^\infty \frac{1}{T(\lambda_{i, d}^{-1},1,p,0)}}_{=:S_p^* <\infty\ \text{by \eqref{eq:tractiff}} }\right] \\
       & \le 1+ T(\varepsilon^{-1},1,p,0) \underbrace{\left[\frac{(C(p)-1)}{T(\lambda_{\max}^{-1},1,p,0)} + S_p^*\right]}_{=:C_2(p)}\\
       & = 1+ T(\varepsilon^{-1},1,p,0) C_2 (p)
\end{align*}
which means that we have strong $T$-tractability, and verifies sufficiency of \eqref{eq:strong_tractiff3}. 

\bigskip

\bigskip
\noindent \textbf{Necessity condition:} \\
Suppose that we have strong 
$T$-tractability. That is, 
\[
\comp(\varepsilon,d)=\Theta (T(\varepsilon^{-1},1,p,0))
\]
for some $p>0$. This, in turn, means that there is a positive constant $K$ such that
\begin{equation}\label{eq_strong_tract_p'3}
\comp(\varepsilon,d)\le K T(\varepsilon^{-1},1,p',0)
\end{equation}
for all $p'>p$. 

\noindent This means for (possibly small) $\delta_p>0$, there exists a constant $K$ such that \eqref{eq_strong_tract_p'3} holds for $p'=p+\delta_p$.

Since the eigenvalues $\lambda_{j,d}$ are non-increasing, we have
\begin{equation}\label{eq:lambda_K_strong3}
\lambda_{\lfloor K T(\varepsilon^{-1},1,p',0)\rfloor +1,d}\le \varepsilon.
\end{equation}
Let
\[
j=j (\varepsilon):= \lfloor K T(\varepsilon^{-1},1,p',0)\rfloor +1.
\]

Suppose we vary $\varepsilon\in (0,1]$, we have that $j=j_d^*, j_d^*+1, j_d^*+2,\ldots$, where 
\[
  j_d^*=\lfloor K T(1,1,p',0)\rfloor +1
\]
Note that
\[
j\le K T(\varepsilon^{-1},1,p',0)+1.
\]
Due to our assumptions, we know that 
$T(\cdot,1, p',0)$ is increasing in the first argument if 
we keep $p'$ fixed. Let $T_{p'}^{-1} (\cdot)$ be the inverse function of $T (\cdot,1, p',0)$ with respect to the first argument. Observe also that $T_{p'}^{-1}$ is strictly increasing. 


Then, 
\[
j \le K T(\varepsilon^{-1},1,p',0)+1
\]
is equivalent to
\[
\frac{j-1}{K} \le T(\varepsilon^{-1},1,p',0).
\]
By the monotonicity of $T_{p'}^{-1}$, the latter inequality holds if and only if
\[
  T_{p'}^{-1} \left(\frac{j-1}{K} \right)
  \le \varepsilon^{-1},
\]
or, equivalently,
\[
 \varepsilon \le \frac{1}{T_{p'}^{-1} \left(\frac{j-1}{K} \right)}.
\]
By using \eqref{eq:lambda_K_strong3}, we derive
\[
 \lambda_{j,d}\le \frac{1}{T_{p'}^{-1} \left(\frac{j-1}{K} \right)}
\]
for all $j\ge j_d^*$, which is again equivalent to 
\[
\lambda_{j,d}^{-1}\ge T_{p'}^{-1} \left(\frac{j-1}{K} \right),
\]
which holds if and only if
\[ 
 T(\lambda_{j,d}^{-1},1, p',0) \ge \frac{j-1}{K}.
\]
This implies, for any $\tau>1$, that 
\[ 
 (T(\lambda_{j,d}^{-1},1, p',0))^\tau \ge \frac{(j-1)^\tau}{K^\tau },
\]
which implies, due to our assumption on $\phi$, 
\[ 
 K_{p',\tau}\,T (\lambda_{j,d}^{-1},1,\tau p',0) \ge \frac{(j-1)^\tau}{K^\tau }.
\]
This yields
\[
\sum_{j=j_d^*}^\infty \frac{1}{T (\lambda_{j,d}^{-1},1, \tau p',0)} \le K_{p',\tau}\, K^\tau \sum_{j=j_d^*}^\infty \frac{1}{(j-1)^\tau}<\infty.
\]

Note, however, that 
\[
 j_d^* = \lfloor K T(1,1,p',0)\rfloor +1 \le \lfloor K T(1,1,\tau p',0)\rfloor +1
\]
Hence, we can find $C(\tau p')$ such that 
\[
 \sum_{j=C(\tau p')}^\infty \frac{1}{T(\lambda_{j,d}^{-1},1, \tau p',0)}<\infty.
\]
This yields \eqref{eq:strong_tractiff3} with $p$ replaced by $\tau p'$, and so we see necessity of $\eqref{eq:strong_tractiff3}$. 



\end{proof}
\bigskip
\begin{theorem}\label{thm_main_strong_tract2} 
A problem is $T$-tractable iff there exist $p,q>0$ and an integer $C(p,q) \ge 0$ such that
\begin{equation} \label{eq:tractiff4}
     \sup_{d \in \naturals} \sum_{i = C(p,q)}^\infty \frac{1}{T(\lambda^{-1}_{i,d},d,p,q)}< \infty.
\end{equation}
%$ \frac{1}{\psi(d,q)}\frac{1}{\phi(\lambda_{i, d}^{-1},p))} $

\end{theorem}

\begin{proof}
    \textbf{Sufficient Condition:}\\ Suppose that \eqref{eq:tractiff4} holds for some $p,q>0$. 
Because $T$ is increasing in its arguments, it follows that the complexity may be equivalently expressed as 
\begin{align*}
    \comp(\varepsilon, d) &= \min \{n \in \natzero : \lambda_{n+1, d} \le \varepsilon\} \\
    &= \min \biggl\{n \in \natzero : \frac{1}{T(\lambda_{n+1, d}^{-1},d,p,q)} \le \frac{1}{T(\varepsilon^{-1},d,p,q)} \biggr\}.
\end{align*}

Since the $\lambda_{i,d}$ are non-increasing it follows that the $\phi(\lambda_{i, d}^{-1},p)$ are non-decreasing, and this holds for any $\tau>1$.  
In particular, for all $n \in \naturals$,
\begin{align*}
    \MoveEqLeft{\lambda_{n+1,d} \le \lambda_{n,d} \le \cdots \le \lambda_{1,d}} \\
    & \implies \frac{1}{T(\lambda_{n+1, d}^{-1},d,p,q)} \le \frac{1}{T(\lambda_{n, d}^{-1},d,p,q))} \le \cdots \le \frac{1}{T(\lambda_{1, d}^{-1},d,p,q)} \\
    & \implies \frac{1}{T(\lambda_{n+1, d}^{-1},d,p.q)} 
    \le \frac 1n \sum_{i=1}^n  \frac{1}{T(\lambda_{i, d}^{-1},d,p,q)} 
    \le \frac 1n \sum_{i=1}^\infty  \frac{1}{T(\lambda_{i, d}^{-1},d,p,q)}.
\end{align*}
Thus, we can conclude that 
\begin{align*}
    n \ge T(\varepsilon^{-1},d,p,q) \sum_{i=1}^\infty \frac{1}{T(\lambda_{i, d}^{-1},d,p,q)}
    & \implies 
  \frac 1n \sum_{i=1}^\infty \frac{1}{T(\lambda_{i, d}^{-1},d,p,q)} \le  \frac{1}{T(\varepsilon^{-1},d,p,q)} \\
   & \implies   \frac{1}{T(\lambda_{n+1, d}^{-1},d,p,q)} \le \frac{1}{T(\varepsilon^{-1},d,p,q)}.
\end{align*}
This implies an upper bound on the complexity of
\begin{align*}
       \MoveEqLeft{\comp(\varepsilon,d)} \\
       &\le 1 + T(\varepsilon^{-1},d,p,q) \sum_{i=1}^\infty \frac{1}{T(\lambda_{i, d}^{-1},d,p,q)} \\       
       & = 1 + T(\varepsilon^{-1},d,p,q) \left [ \sum_{i=1}^{C(p,q)-1} \frac{1}{T(\lambda_{i, d}^{-1},d,p,q)}  
       + \sum_{i=C(p,q)}^\infty \frac{1}{T(\lambda_{i, d}^{-1},d,p,q)} \right ] \\
       & \le 1+ T(\varepsilon^{-1},d,p,q) \underbrace{\frac{C(p,q)-1}{T(\lambda_{\max}^{-1},d,p,q)}}_{=:C_1(p,q)}
       +  T(\varepsilon^{-1},d,p,q) 
        \qquad \text{by \eqref{eq:tractiff4}} \\
        & \le 1+T(\varepsilon^{-1},d,p,q) \bigl[C_1(p,q) +1\bigr].
\end{align*}
It follows that we have $T$-tractability, which shows sufficiency of \eqref{eq:tractiff3}. \\

\textbf{Necessity condition:}\\
Suppose that we have 
$T$-tractability with parameters $p$ and $q$. That is, 
\[
\comp(\varepsilon,d)=\Theta (T(\varepsilon^{-1},d,p,q))
\]
for some $p,q>0$. This, in turn, means that there is a positive constant $K$ such that
\begin{equation}\label{eq_tract_p'q'4}
\comp(\varepsilon,d)\le K T(\varepsilon^{-1},d,p',q')
\end{equation}
for all $p'>p$ and $q'>q$. 

Hence, for (possibly small) $\delta_p>0$ and $\delta_q>0$, there exists a constant $K$ such that \eqref{eq_tract_p'q'4} holds for $p'=p+\delta_p$ 
and $q'=q+\delta_q$.

We know that
\[
\lambda_{\comp(\varepsilon,d)+1,d}\le \varepsilon.
\]
Since the eigenvalues $\lambda_{j,d}$ are non-increasing, we have
\begin{equation}\label{eq:lambda_K4}
\lambda_{\lfloor K T(\varepsilon^{-1},d,p',q)\rfloor +1,d}\le \varepsilon.
\end{equation}
Let
\[
j=j (\varepsilon):= \lfloor K T(\varepsilon^{-1},d,p',q')\rfloor +1.
\]
If we vary $\varepsilon\in (0,1]$, we see that $j=j_d^*, j_d^*+1, j_d^*+2,\ldots$, where 
\[
  j_d^*=\lfloor K T(1,d,p',q')\rfloor +1
\]
Note that
\[
j\le K T(\varepsilon^{-1},d,p',q') +1.
\]
Due to our assumptions, we know that 
$T(\cdot,d,p',q')$ is increasing in the first argument if 
we keep $p'$ and $q'$ fixed. Let $T_{p',q'}^{-1} (\cdot)$ be the inverse function of $T(\cdot,d,p',q')$ with respect to the first argument. Note that also $T_{p',q'}^{-1}$ is strictly increasing. 

Then, 
\[
j \le K T(\varepsilon^{-1},d,p',q') +1
\]
is equivalent to
\[
\frac{j-1}{K } \le T(\varepsilon^{-1},d,p',q').
\]
By the monotonicity of $T_{p',q'}^{-1}$, the latter inequality holds if and only if
\[
  T_{p',q'}^{-1} \left(\frac{j-1}{K} \right)
  \le \varepsilon^{-1},
\]
or, equivalently,
\[
 \varepsilon \le \frac{1}{T_{p',q'}^{-1} \left(\frac{j-1}{K} \right)}.
\]
By using \eqref{eq:lambda_K4}, we derive
\[
 \lambda_{j,d}\le \frac{1}{T_{p',q'}^{-1} \left(\frac{j-1}{K } \right)}
\]
for all $j\ge j_d^*$, which is again equivalent to 
\[
\lambda_{j,d}^{-1}\ge T_{p',q'}^{-1} \left(\frac{j-1}{K} \right),
\]
which holds if and only if
\[ 
T(\lambda_{j,d}^{-1},d, p',q') \ge \frac{j-1}{K}.
\]
This implies, for any $\tau>1$, that 
\[ 
 (T(\lambda_{j,d}^{-1},d, p',q'))^\tau \ge \frac{(j-1)^\tau}{K^\tau },
\]
which yields
\[ 
 K_{p',q',\tau}\, T (\lambda_{j,d}^{-1},d,\tau p',\tau q') \ge (T(\lambda_{j,d}^{-1},d, p',q'))^\tau 
 \ge \frac{(j-1)^\tau}{K^\tau}.
\]
This then implies
\[
\sum_{j=j_d^*}^\infty \frac{1}{T(\lambda_{j,d}^{-1},d, \tau p',\tau q')} 
\le K_{p',q',\tau}\, K^\tau \sum_{j=j_d^*}^\infty \frac{1}{(j-1)^\tau}<\infty.
\]
Note now that
\[
 j_d^* = \lfloor K T(1,1,p',q')\rfloor +1 \le \lfloor KT(1,1,\tau p',\tau q')\rfloor +1.
\]
So, there exists $C(\tau p',\tau q')$ such that 
\[
 \sum_{j=C(\tau p',\tau q')}^\infty \frac{1}{T(\lambda_{j,d}^{-1},d, \tau p',\tau q')}<\infty.
\]
This yields \eqref{eq:tractiff} with $p$ replaced by $\tau p'$ and $q$ replaced by $\tau q'$, and so we see the necessity of $\eqref{eq:tractiff4}$. 

\end{proof}
}
\bigskip

\fred{We need to check the proof for 
\begin{itemize}
\item correctness
\item exponents
\item whether the additional assumptions are too restrictive and can be generalized
\item whether we need $\tau$ in the assumption
\item Do we need non-decreasing or strictly increasing? 
\end{itemize}}

\end{document}
