%Introduction to generalized tracability
\documentclass[11pt,compress,xcolor={usenames,dvipsnames},aspectratio=169]{beamer}
%\documentclass[xcolor={usenames,dvipsnames},aspectratio=169]{beamer} %slides and 
%notes
\usepackage[T1]{fontenc}
\usepackage{tgadventor} %Font found at https://tug.org/FontCatalogue/
%\usepackage{newpxtext}
\usepackage[euler-digits,euler-hat-accent]{eulervm}

\usepackage{amsmath,
	amssymb,
	datetime,
	mathtools,
	bbm,
	%mathabx,
	array,
	booktabs,
	xspace,
	multirow,
	calc,
	colortbl,
	siunitx,
 	graphicx}
\usepackage[usenames]{xcolor}
\usepackage[giveninits=false,backend=biber,style=nature, maxcitenames =10, mincitenames=9]{biblatex}
\addbibresource{FJHown23.bib}
\addbibresource{FJH23.bib}
\usepackage{media9}
\usepackage[autolinebreaks]{mcode}
\usepackage[tikz]{mdframed}


\usetheme{FJHSlimNoFoot169}
\setlength{\parskip}{2ex}
\setlength{\arraycolsep}{0.5ex}


\DeclareMathOperator{\COMP}{COMP}
\DeclareMathOperator{\COST}{COST}
\DeclareMathOperator{\ALG}{ALG}
\DeclareMathOperator{\SOL}{SOL}
\DeclareMathOperator{\APP}{APP}
\DeclareMathOperator{\ERR}{ERR}
\DeclareMathOperator{\AVG}{AVG}
\DeclareMathOperator{\INT}{INT}
\DeclareMathOperator{\LIN}{LINEAR}
\DeclareMathOperator{\BAD}{BAD}
%\DeclareMathOperator{\opt}{opt}
\newcommand{\dataN}{\bigl(\hf(\vk_i)\bigr)_{i=1}^n}
\newcommand{\dataNj}{\bigl(\hf(\vk_i)\bigr)_{i=1}^{n_j}}
\newcommand{\dataNjd}{\bigl(\hf(\vk_i)\bigr)_{i=1}^{n_{j^\dagger}}}
\newcommand{\ERRN}{\ERR\bigl(\dataN,n\bigr)}
\newcommand{\otod}{\ensuremath{1\mkern-4mu : \mkern-2mu d}}


%\DeclareMathOperator{\app}{app}

\providecommand{\HickernellFJ}{H.\xspace}


\renewcommand{\OffTitleLength}{-7ex}
\setlength{\FJHThankYouMessageOffset}{-8ex}
\title{Generalized Tractability}
\author[]{Fred J. Hickernell, Peter Kritzer, Onyekachi Osisiogu (?)}
\institute{Department of Applied Mathematics \qquad
	Center for Interdisciplinary Scientific Computation \\
	Illinois Institute of Technology \qquad
	\href{mailto:hickernell@iit.edu}{\url{hickernell@iit.edu}} \qquad
	\href{http://mypages.iit.edu/~hickernell}{\url{mypages.iit.edu/~hickernell}}}

\thanksnote{}
	
%\event{Happy Fred}
\date[]{ revised \today}

\input FJHDef.tex



\begin{document}
	\everymath{\displaystyle}

\frame{\titlepage}

%%%%%%%%%%%%%%%%%%%%%%%%%%%%%%%%%%%%%%%%%%
%%%%%%%%%%%%%%%%%%%%%%%%%%%%%%%%%%%%%%%%%%
\section{Background}
%%%%%%%%%%%%%%%%%%%%%%%%%%%%%%%%%%%%%%%%%%
%%%%%%%%%%%%%%%%%%%%%%%%%%%%%%%%%%%%%%%%%%

\begin{frame}{\only<1-3>{Cost of the Algorithm, Complexity of the Problem}\only<4->{$d$-Dimensional Problems}}
\vspace{-6ex}
\only<4->{\vspace{-6ex}}
   \begin{gather*}
   \text{\alert{integration \only<1-3>{problem}}\quad} \only<1-3>{\SOL}\only<4->{\SOL_{\alert{d}}}(f) := \only<1-3>{\int_{0}^1}\only<4->{\int_{\alert{[0,1]^d}}} f(x) \, \dif x = \,?,  \quad f \in \only<1-3>{\{f : \norm[2]{f''} \le R \}}\only<4->{\{f : \norm[2]{\partial^{\valpha}f} \le R \; \forall \norm[\infty]{\valpha} \le 2 \}} =: \only<1-3>{\cb}\only<4->{\cb_{\alert{d}}}\\
   \text{\alert{\only<4->{tensor product }trapezoidal rule}\quad} \only<1-3>{\APP}\only<4->{\APP_{\alert{d}}}(f,n) \only<1-3>{:= \frac 1n \biggl( \frac12 f(0) + f(1/n) + \dots + f(1-1/n) + \frac 12 f(1) \biggr)} \\
   \text{\alert{error}\quad}\only<1-3>{\ERR}\only<4->{\ERR_{\alert{d}}}(f,n) = \abs{\only<1-3>{\SOL}\only<4->{\SOL_{\alert{d}}}(f) - \only<1-3>{\APP}\only<4->{\APP_{\alert{d}}}(f,n)} \only<1-3>{\le \frac{R}{8n^2}}\only<4->{ = \Theta(n^{-2/\alert{d}})} \\
   \uncover<2->{\text{\alert{algorithm}\quad} \only<1-3>{\ALG}\only<4->{\ALG_{\alert{d}}}(f,\varepsilon) = \only<1-3>{\APP}\only<4->{\APP_{\alert{d}}}(f,\only<1-3>{n_\varepsilon}\only<4>{n_{\varepsilon,d}}), \qquad \only<1-3>{n_\varepsilon}\only<4->{n_{\varepsilon,d}}\only<1-3>{ = \sqrt{\frac{R}{8\varepsilon}}} \quad \text{ensures that}\\
   \text{\alert{error criterion}\quad} \abs{\only<1-3>{\SOL}\only<4->{\SOL_{\alert{d}}}(f) - \only<1-3>{\ALG}\only<4->{\ALG_{\alert{d}}}(f,\varepsilon)} \le \varepsilon \quad \forall f \in \cb\only<4->{\alert{_d}} \\
   \text{\alert{information cost} \quad}\only<1-3>{\COST}\only<4->{\COST_{\alert{d}}}(\only<1-3>{\SOL}\only<4->{\SOL_{\alert{d}}},
   \only<1-3>{\cb}\only<4->{\cb_{\alert{d}}},
   \only<1-3>{\ALG}\only<4->{\ALG_{\alert{d}}},\varepsilon) = \only<1-3>{n_\varepsilon}\only<4->{n_{\varepsilon,d}} = \Theta (\varepsilon^{-\only<1-3>{1}\only<4->{\alert{d}}/2}) \text{ as }\varepsilon \to 0}
   \end{gather*}
   \only<1-3>{\vspace{-4ex}}\only<4->{\vspace{-6ex}
   
   \hfill \hfill The \alert{curse} of dimensionality
   \vspace{-2ex}}
   \uncover<3->{\begin{multline*}
          \text{\alert{computational complexity} \quad}\only<1-3>{\COMP}\only<4->{\COMP_{\alert{d}}}(\only<1-3>{\SOL}\only<4->{\SOL_{\alert{d}}},
          \only<1-3>{\cb}\only<4->{\cb_{\alert{d}}}
          ,\varepsilon) \\
          = \text{information cost of \alert{best} possible algorithm} = \Theta (\only<4->{C_{\alert{d}}
          } \varepsilon^{-1/2}) \text{ as }\varepsilon \to 0
   \end{multline*}}
   \only<5>{What about as \alert{$d \to \infty$?}, the question of \alert{tractability}}
   
\vspace{-7ex}
   \only<3>{\begin{equation*}
          g(\varepsilon) = \Theta (\varepsilon^{-p}) \text{ means }
          \lim_{\varepsilon \to 0} g(\varepsilon) \varepsilon^{q}
          = \begin{cases} 0, & q > p  \\ \infty, & q < p\end{cases}
   \end{equation*}}
   

\end{frame}

\section{Harder and Easier}
\begin{frame}{Linear Problem with Linear Information}
    \vspace{-7ex}
   \begin{gather*}
   \text{\alert{problem}}\quad \SOL_d : \cf_d \to \cg_d,  \qquad \cb_d : = \{f : \norm[\cf_d]{f} \le 1 \} \\
   \SOL_d^*\SOL_d : \cf_d \to \cf_d \text{ has eigenvalues and orthonormal eigenvectors } \\
\lambda_{\max} \ge \lambda_{1,d} \ge \lambda_{2,d} \ge \cdots, \qquad u_{1,d}, u_{2,d}, \ldots, \\
   \text{\alert{approximation}\quad} \APP_d(f,n) = \sum_{i=1}^n \SOL(u_{i,d}) \ip[\cf_d]{f}{u_{i,d}}, \\
   \text{\alert{error}\quad}\ERR_d(f,n) = \norm[\cg]{\SOL_d(f) - \APP_d(f)} \le \sqrt{\lambda_{n+1, d}} \\
   \text{\alert{algorithm}\quad} \ALG_d(f,\varepsilon) = \APP_d(f,n_\varepsilon), \qquad n_{\varepsilon,d} = \min \{n \in \natzero : \lambda_{n+1, d} \le \varepsilon^2\} \quad \text{ensures}\\
   \text{\alert{error criterion}\quad} \norm[\cg]{\SOL_d(f) - \ALG_d(f,\varepsilon)} \le \varepsilon \quad \forall f \in \cb_d \\
   \text{\alert{information cost} \quad} \COST_d(\SOL_d,\cb_d,\ALG_d,\varepsilon) = n_{\varepsilon,d} = \COMP(\SOL_d,\cb_d,\varepsilon)
   \end{gather*}
   
   \vspace{-3ex}
   The question is $\COMP_d(\SOL_d,\cb_d,\varepsilon) = \Theta(\varepsilon^{-p} d^q)$ (tractable)? $\Theta(\varepsilon^{-p})$ (strongly tractable)? $\Theta\bigl(\log(1 + \varepsilon^{-p} ) d^q \bigr)$ (exponentially  tractable)?
   \hfill \hfill as $\varepsilon \to 0$, $d \to \infty$
   

\end{frame}


\section{A General Approach to Tractability}
\begin{frame}{Moving Towards a Theorem}
    \vspace{-6ex}
    \[
    \text{\alert{computational complexity}\quad} \COMP(\SOL_d,\cb_d,\varepsilon) = \min \{n \in \natzero : \lambda_{n+1, d} \le \varepsilon^2\} 
    \]
    
    \vspace{-3ex}
   The question is $\COMP_d(\SOL_d,\cb_d,\varepsilon) = \Theta(\varepsilon^{-p} d^q)$ (tractable)? $\Theta(\varepsilon^{-p})$ (strongly tractable)? $\Theta\bigl(\log(1 + \varepsilon^{-p} ) d^q \bigr)$ (exponentially  tractable)?
   \hfill \hfill as $\varepsilon \to 0$, $d \to \infty$
   
   Arguments seem to be similar for these cases and give an equivalent condition on the sums of functions of the $\lambda_{i,d}$
 
   Let let $\phi : (0,\infty) \times [0,+\infty) \to (0,\infty)$ and $\psi : \naturals  \times [0,+\infty) \to (0,\infty)$ be positive functions that are increasing in both arguments, e.g., 
\[
(x,p) \mapsto x^{p}, \qquad (x,q) \mapsto \log(1+ x^q)
\]

\vspace{-3ex}
\alert{Definition.} A problem is $\phi(\cdot,p)$-$\psi(\cdot, q)$-tractable iff $\COMP_d(\SOL_d,\cb_d,\varepsilon) = \Theta\bigl(\phi(\varepsilon^{-1},p) \psi(d,q)  \bigr)$



\end{frame}

\begin{frame}{Possible Theorem}
\begin{theorem} A problem is $\phi(\cdot,p)$-$\psi(\cdot, q)$-tractable iff(?) there exists a  $C : (0,\infty)^2 \to (0,\infty)$
\begin{equation*}
    \sup_{d \in \naturals} \frac{1}{\psi(d,q')} \sum_{i = C(p',q')}^\infty \frac{1}{\phi(\lambda_{i, d}^{-1/2},p')} \le 1 \qquad \forall p' > p, \ q' > q
\end{equation*}
\end{theorem}

A sketch of one side of the proof is in the notes.

\end{frame}

\begin{frame}{Start of Proof}
\begin{proof}
\vspace{-3ex}
Fix $p' > p$ and $q' > q$.  Because $\phi$ is increasing in its arguments, it follows that

\vspace{-6ex}
\begin{align*}
    \comp(\varepsilon, d) &= \min \{n \in \natzero : \lambda_{n+1, d} \le \varepsilon^2\} \\
    &= \min \biggl\{n \in \natzero : \frac{1}{\phi(\lambda_{n+1, d}^{-1/2},p')} \le \frac{1}{\phi(\varepsilon^{-1},p')} \biggr\}.
\end{align*}

\vspace{-4ex}
Since the $\lambda_{i,d}$ then the $\phi(\lambda_{i, d}^{-1/2},p')$ are also non-increasing.  For all $n \in \naturals$,

\vspace{-6ex}
\begin{align*}
    \MoveEqLeft{\lambda_{n+1,d} \le \cdots \le \lambda_{1,d}} \\
    & \implies \frac{1}{\phi(\lambda_{n+1, d}^{-1/2},p')}  \le \cdots \le \frac{1}{\phi(\lambda_{1, d}^{-1/2},p')} \\
    & \implies \frac{1}{\phi(\lambda_{n+1, d}^{-1/2},p')} 
    \le \frac 1n \sum_{i=1}^n  \frac{1}{\phi(\lambda_{i, d}^{-1/2},p')}
\end{align*}
\end{proof}
\end{frame}

\begin{frame}{Continuation of Start of Proof}
\begin{proof}
\vspace{-3ex}
Thus, we can conclude that 
\begin{align*}
    n \ge \phi(\varepsilon^{-1},q') \sum_{i=1}^\infty \frac{1}{\phi(\lambda_{i, d}^{-1/2},p')}
    & \implies 
  \frac 1n \sum_{i=1}^\infty \frac{1}{\phi(\lambda_{i, d}^{-1/2},p')} \le  \frac{1}{\phi(\varepsilon^{-1},p')} \\
   & \implies   \frac{1}{\phi(\lambda_{n+1, d}^{-1/2},p')} \le \frac{1}{\phi(\varepsilon^{-1},p')}.
\end{align*}
\end{proof}
\end{frame}

\begin{frame}{Continuation of Start of Proof}
\begin{proof}
\vspace{-3ex}
This implies an upper bound on the complexity of

\vspace{-6ex}
\begin{align*}
       \comp(\varepsilon,d)
       &\le \phi(\varepsilon^{-1},q') \sum_{i=1}^\infty \frac{1}{\phi(\lambda_{i, d}^{-1/2},p')} \\
       & = \phi(\varepsilon^{-1},p') \left [ \sum_{i=1}^{C(p',q')-1} \frac{1}{\phi(\lambda_{i, d}^{-1/2},p')}  + \sum_{i=C(p',q')}^\infty \frac{1}{\phi(\lambda_{i, d}^{-1/2},p')} \right ] \\
       & \le \underbrace{\phi(\varepsilon^{-1},p') \frac{C(p',q')-1}{\phi(\lambda_{\max}^{-1/2},p')}}_{C'(p',q')}
       +  \phi(\varepsilon^{-1},p') \psi(d,q') \le \Theta\bigl(\phi(\varepsilon^{-1},p) \psi(d,q)  \bigr)
\end{align*}
This verifies the sufficient condition.
\end{proof}

\end{frame}
\end{document}

